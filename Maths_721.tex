\documentclass[11pt, notitlepage]{article}

\usepackage[english]{babel}
\usepackage[utf8x]{inputenc}
\usepackage{amsmath}
\usepackage{amssymb}
\usepackage{mathtools}
\usepackage{amssymb}
\usepackage{amsfonts}
\usepackage{mathdots}
\usepackage{multicol}
\usepackage{array}
\usepackage{cool}
\usepackage{parskip}
\usepackage{tikz}
\usetikzlibrary{automata, arrows.meta, chains}
\usepackage[a4paper]{geometry}
\usepackage{tensor}

\usepackage{color}
\usepackage{siunitx}
\usepackage{hyperref}
\usepackage{amsthm}
\usepackage{enumitem}
\usepackage{tikz-cd}
\usepackage{mathrsfs}
\usepackage{natbib}
\usepackage{fancyhdr}
\usepackage[nottoc]{tocbibind}
\usepackage{theoremref}


\pagestyle{fancy}
\fancyhf{}
\fancyhead[R]{\textit{\rightmark}}
\fancyfoot[C]{Page \thepage}

\usepackage{microtype}
\DisableLigatures[f]{encoding = *, family = *}


\newcommand\mapsfrom{\mathrel{\reflectbox{\ensuremath{\mapsto}}}}



\renewcommand{\footrulewidth}{0.5pt}

\bibliographystyle{alpha}
\addto\captionsenglish{\renewcommand{\bibname}{References}}

%\usepackage[backref=page,pagebackref=true,linkcolor = blue,citecolor = red]{hyperref}
%\usepackage[backref=page]{backref}


\usepackage{graphicx}
\DeclareGraphicsExtensions{.pdf,.png,.jpg}

\numberwithin{equation}{section}

\theoremstyle{plain}
\newtheorem{theorem}{Theorem}[section]
\newtheorem{corollary}{Corollary}[theorem]
\newtheorem{lemma}[theorem]{Lemma}
\newtheorem{proposition}[theorem]{Proposition}


\theoremstyle{definition}
\newtheorem{definition}[theorem]{Definition}
\newtheorem{problem}{Problem}
\newtheorem{remarkx}{Remark}
\newenvironment{remark}
	{\pushQED{\qed}\renewcommand{\qedsymbol}{$\blacklozenge$}\remarkx}
	{\popQED\endremarkx}
	
\newtheorem{examplex}[theorem]{Example}
\newenvironment{example}
	{\pushQED{\qed}\renewcommand{\qedsymbol}{$\blacktriangleleft$}\examplex}
	{\popQED\endexamplex}


% Syntax for below is \cv{a_1,a_2,...,a_n}.
% Creates a square bracketed column vector.
\makeatletter
\newcommand{\cv}[2][r]{%
  \gdef\@VORNE{1}
  \left[\hskip-\arraycolsep%
    \begin{array}{#1}\vekSp@lten{#2}\end{array}%
    \hskip-\arraycolsep\right]}

\def\vekSp@lten#1{\xvekSp@lten#1,vekL@stLine,}
\def\vekL@stLine{vekL@stLine}
\def\xvekSp@lten#1,{\def\temp{#1}%
  \ifx\temp\vekL@stLine
  \else
    \ifnum\@VORNE=1\gdef\@VORNE{0}
    \else\@arraycr\fi%
    #1%
    \expandafter\xvekSp@lten
  \fi}
\makeatother

\makeatletter
\renewcommand*\env@matrix[1][\arraystretch]{%
  \edef\arraystretch{#1}%
  \hskip -\arraycolsep
  \let\@ifnextchar\new@ifnextchar
  \array{*\c@MaxMatrixCols c}}
\makeatother

\newcommand{\mcl}{\mathcal}

\newcommand{\V}{\mathcal{V}}
\newcommand{\calL}{\mathcal{L}}
\newcommand{\A}{\mathbb{A}}
\newcommand{\R}{\mathbb{R}}
\newcommand{\C}{\mathbb{C}}
\newcommand{\Z}{\mathbb{Z}}
\newcommand{\Q}{\mathbb{Q}}
\newcommand{\N}{\mathbb{N}}
\newcommand{\F}{\mathbb{F}}
\newcommand{\T}{\mathbb{T}}
\renewcommand{\E}{\mathbb{E}}
\newcommand{\K}{\mathbb{K}}
\newcommand{\B}{\mathbb{B}}
\newcommand{\sph}{\mathbb{S}}
\newcommand{\Halfspace}{\mathbb{H}}
\renewcommand{\P}{\mathbb{P}}
\newcommand{\inner}[2]{\left\langle #1,#2 \right\rangle}
\newcommand{\tbasis}[1]{\frac{\partial}{\partial #1}}
\newcommand{\extp}[1]{\bigwedge\nolimits^{#1}}



\newcommand{\abs}[1]{\left\lvert#1\right\rvert}
\newcommand{\norm}[1]{\left\lVert#1\right\rVert}
\DeclareMathOperator{\proj}{proj}
\DeclareMathOperator{\cis}{cis}
\let\Arg\relax
\DeclareMathOperator{\Arg}{Arg}
\DeclareMathOperator{\col}{col}
\DeclareMathOperator{\rank}{rk}
\DeclareMathOperator{\row}{row}
\DeclareMathOperator{\nul}{null}
\DeclareMathOperator{\spn}{span}
\DeclareMathOperator{\Mat}{Mat}
\let\hom\relax
\DeclareMathOperator{\hom}{Hom}
\DeclareMathOperator{\epi}{epi}
\DeclareMathOperator{\Sym}{Sym}
\DeclareMathOperator{\GL}{GL}
\DeclareMathOperator{\SL}{SL}
\DeclareMathOperator{\sgn}{sgn}
\DeclareMathOperator{\Aff}{Aff}
\DeclareMathOperator{\lcm}{lcm}
\DeclareMathOperator{\Isom}{\textbf{Isom}}
\DeclareMathOperator{\gen}{gen}
\DeclareMathOperator{\im}{im}
\DeclareMathOperator{\Aut}{Aut}
\DeclareMathOperator{\Inn}{Inn}
\DeclareMathOperator{\Sub}{\textbf{Sub}}
\DeclareMathOperator{\ind}{ind}
\DeclareMathOperator{\exterior}{Ext}
\DeclareMathOperator{\interior}{Int}
\DeclareMathOperator{\identity}{Id}
\DeclareMathOperator{\orthogonal}{O}
\DeclareMathOperator{\supp}{supp}
\DeclareMathOperator{\real}{Re}
\DeclareMathOperator{\imagine}{Im}
\DeclareMathOperator{\Ind}{Ind}
\DeclareMathOperator{\SRW}{SRW}
\DeclareMathOperator{\Markov}{Markov}
\DeclareMathOperator{\tr}{tr}
\DeclareMathOperator{\Poisson}{Poisson}
\DeclareMathOperator{\Geometric}{Geometric}
\DeclareMathOperator{\Exponential}{Exp}
\DeclareMathOperator{\cl}{cl}
\DeclareMathOperator{\Diff}{Diff}
\DeclareMathOperator{\ad}{ad}
\DeclareMathOperator{\Ad}{Ad}
\DeclareMathOperator{\unitary}{U}
\DeclareMathOperator{\End}{End}
\DeclareMathOperator{\inprod}{\lrcorner}
\DeclareMathOperator{\rot}{rot}
\DeclareMathOperator{\grad}{grad}
\DeclareMathOperator{\first}{I}
\DeclareMathOperator{\second}{II}
\DeclareMathOperator{\third}{III}
\DeclareMathOperator{\projection}{pr}
\DeclareMathOperator{\coker}{coker}
\DeclareMathOperator{\vol}{vol}
\DeclareMathOperator{\SO}{SO}


\def\*#1{\mathbf{#1}}
\newcommand{\veps}{\varepsilon}
\newcommand{\vphi}{\varphi}
\newcommand{\normsub}{\unlhd}
\newcommand{\snormsub}{\lhd}
\title{Maths 721 Notes}
\date{2020}

\begin{document}

\maketitle
\tableofcontents
\vspace{5mm}


% -------------------------------------------------------------------
% LECTURE 1
% -------------------------------------------------------------------

\pagebreak

\section{Lecture 1}


In the first half of this course we will cover three main topics:
\begin{itemize}
    \item representations;
    \item modules;
    \item characters.
\end{itemize}
We will further see that representations and modules are essentially the same, and that modules and characters are essentially the same; and hence all three are essentially the same.

From now on $G$ is a group.



\subsection{Representations}


\begin{definition}
A \textbf{representation} of a group $G$ over a field $F$ is a group homomorphism from $G$ to $\GL(n,F)$, where $n$ is the \textbf{degree} of the representation. 
\end{definition}

Explicitly, a representation is a function $\rho : G \to \GL(n,F)$ such that for all $g,h \in G$;
\begin{enumerate}[label=(\roman*)]
    \item $(gh)\rho = (g\rho)(h\rho)$,
    \item $1_G\rho = I_n$,
    \item $g^{-1}\rho = (g\rho)^{-1}$.
\end{enumerate}
Note the use of the (incredibly shit) postfix function notation.

\begin{example} \thlabel{d4}
Take $D_4$, the Dihedral group of order 8. It has the following group presentations
\begin{align*}
    D_4 &= \langle a,b \mid a^4 = 1, b^2 = 1, a^b = a^{-1} \rangle\\
    &\cong \langle (1 \; 2 \; 3 \; 4), (1 \; 4)(2 \; 3) \rangle,
\end{align*}
where $a^b = bab^{-1}$ is conjugation of $a$ by $b$. By defining the matrix subgroup
\[
    H = \left\langle A = \begin{bmatrix*}[r]
        0 & 1\\
        -1 & 0
    \end{bmatrix*}, B = \begin{bmatrix*}[r]
        1 & 0\\
        0 & -1
    \end{bmatrix*}\right\rangle
\]
and defining $\rho : D_4 \to \GL(n,F)$ where $F = \R, \C$, by $a \mapsto A$ and $b \mapsto B$, and
$a^ib^j \mapsto A^iB^j$ for $0 \le i \le 3$, and $0 \le j \le 1$. Hence we have $\rho$ is a representation of $D_4$ over $F$.
\end{example}

\begin{example}
Take $\Q_8$ the Quaternion group of order 8, which has the following group presentations
\begin{align*}
    \Q_8 &= \langle a,b \mid a^4 = 1, a^2 = b^2, a^b = a^{-1}\rangle\\
    &\cong \langle \bar a = (1 \; 6 \; 2 \; 5)(3 \; 8 \; 4 \; 7), \bar b = (1 \; 4 \; 2 \; 3)(5 \; 7 \; 6 \; 8)\rangle
\end{align*}
Define
\[
    H = \left\langle A = \begin{bmatrix*}[r]
        i & 0\\
        0 & -i
    \end{bmatrix*}, B = \begin{bmatrix*}[r]
        0 & 1\\
        -1 & 0
    \end{bmatrix*}\right\rangle \subset \GL(2,\C).
\]
Then $\rho : \Q_8 \to \GL(2,\C)$ defined by $a^kb^\ell \mapsto A^kB^\ell$ is a group representation of $\Q_8$ over $\C$ of degree 2.
\end{example}


\begin{definition}
Let $G$ be a group and define
\begin{align*}
    \rho : G &\to \GL(n,F)\\
    g\rho &= I_n
\end{align*}
for all $g \in G$. Then $\rho$ is a representation, called the \textbf{trivial representation} of arbitrary of degree.
\end{definition}

It follows from the trivial representation that any group $G$ has a representation of an arbitrary degree.

Let $\rho : G \to \GL(n,F)$ be a group homomorphism, and take $T \in \GL(n,F)$. Then
\[
    (T^{-1}AT)(T^{-1}BT) = T^{-1}(AB)T.
\]
Thus, given $\rho$ define $\sigma$ such that
\[
    g\sigma = T^{-1}(g\rho)T
\]
for all $g \in G$. As for all $g,h \in G$, one has
\begin{align*}
    (gh)\sigma &= T^{-1}\big((gh)\rho\big)T\\
    &= T^{-1}(g\rho)(h\rho)T\\
    &= T^{-1}(g\rho)TT^{-1}(h\rho)T\\
    &= (g\sigma)(h\sigma),
\end{align*}
and so $\sigma$ is a group homomorphism; and hence a representation.

\begin{definition}
Define
\[
    \rho : G \to \GL(m,F), \qquad \sigma : G \to \GL(n,F)
\]
to both be representation of $G$ over $F$. We say that \textbf{$\rho$ is equivalent to $\sigma$} if $n=m$ and there exists $T \in \GL(n,F)$ such that $g\sigma = T^{-1}(g\rho)T$.
\end{definition}

\begin{proposition}
Equivalence of representations is an equivalence relation.
\end{proposition}

\begin{proof}
Reflexivity is clear by taking $T = I_n$. For symmetry, take $T$ to be its inverse. For transitivity, if
\[
    g\sigma = T^{-1}(g\rho)T, \qquad g\rho = S^{-1}(g\eta)S,
\]
then
\[
    g\sigma = (ST)^{-1}(g\eta)(ST).
\]
\end{proof}

\begin{definition}
Define the \textbf{kernel} of the representation $\rho : G \to \GL(n,F)$ as $\ker \rho = \{g \in G \mid g\rho = I_n\}$.
\end{definition}

\begin{proposition}
The kernel of a representation of $G$ is a normal subgroup of $G$; i.e. $\ker \rho \lhd G$.
\end{proposition}

\begin{proof}
Suppose $g \in \ker \rho$ and $h \in G$ is arbitrary. Then
\[
    (hgh^{-1})\rho = (h\rho)(g\rho)(h^{-1}\rho) = (h\rho)I_n(h\rho)^{-1} = (h\rho)(h\rho)^{-1} = I_n,
\]
and so $hgh^{-1} \in \ker \rho$. As $\ker \rho$ is closed under conjugation, it is a normal subgroup of $G$.
\end{proof}

\begin{definition}
We say $\rho$ is a \textbf{faithful} representation of $G$ if $\ker \rho = \{1_G\}$. 
\end{definition}

\begin{example}
For the trivial representation $\rho : G \to \GL(n,F)$ with $g \mapsto I_n$ for all $g \in G$, we have $\ker \rho = G$. Hence the representation is not faithful.
\end{example}

\begin{lemma}
Suppose $G$ is a finite group, and $\rho$ is a representation of $G$ over $F$. Then $\rho$ is faithful if, and only if, $\im \rho \cong G$.
\end{lemma}

\begin{proof}
Immediate from the first isomorphism theorem.
\end{proof}



\subsection{$FG$-Modules}



Suppose $G$ is a group, and $F = \R,\C$. Given $\rho : G \to \GL(n,F)$, with $V = F^n$, let $v = (\lambda_1,\dots,\lambda_n) \in V$ for $\lambda_i \in F$ be a row vector. Moreover, note that $g\rho$ is an $n \times n$ matrix for all $g \in G$. Thus, we have $v \cdot (g\rho) \in V$, and satisfies the following properties:
\begin{enumerate}[label=(\roman*)]
    \item $v \cdot \big( (gh)\rho \big) = v \cdot (g\rho)(h\rho)$;
    \item $v \cdot (1_G\rho) = v$;
    \item $(\lambda v) \cdot (g\rho) = \lambda \big( v \cdot (g\rho) \big)$;
    \item $(u + v) \cdot (g\rho) = u \cdot (g\rho) + v \cdot (g\rho)$.
\end{enumerate}

We often will omitted the $\cdot$ in the operation, and write $v(a\rho)$ for $v \cdot (a\rho)$.

\begin{example}
Recall $D_4$ and its given presentation from a previous example. We have
\[
    a\rho = \begin{bmatrix*}[r]
        0 & 1\\
        -1 & 0
    \end{bmatrix*}, \qquad b\rho = \begin{bmatrix*}[r]
        1 & 0\\
        0 & -1
    \end{bmatrix*}.
\]
If $v = (\lambda_1,\lambda_2)$, then we have
\[
    v(a\rho) = (-\lambda_2,\lambda_1), \qquad v(b\rho) = (\lambda_1,-\lambda_2).
\]
\end{example}


\begin{definition}
Let $V$ a vector space over the field $F = \R,\C$, and let $G$ be a group. We say $V$ is a \textbf{$FG$-module} if a multiplication $v \cdot g$ for $v \in V$, and $g \in G$ is defined such that:
\begin{enumerate}[label=(\roman*)]
    \item $v \cdot g \in V$;
    \item $v \cdot (gh) = (v \cdot g) \cdot h$;
    \item $v \cdot 1_G = v$;
    \item $(\lambda v) \cdot g = \lambda(v \cdot g)$;
    \item $(u + v) \cdot g = u \cdot g + v \cdot g$.
\end{enumerate}
\end{definition}

This generalises the previous discussion from a matrix group to an arbitrary group.

Note that properties (i), (iv), and (v) imply that the map $v \mapsto v \cdot g$ is an endomorphism of $V$ (a linear map from $V$ to itself).

\begin{definition}
Suppose $V$ is an $FG$-module and $B$ is a basis for $V$. For $g \in G$, let $[g]_B$ denote the matrix of the endomorphism $v \mapsto v \cdot g$ of $V$ relative to the basis $B$.
\end{definition}




















% -----------------------------------------------------------
% LECTURE 2
% -----------------------------------------------------------



\section{Lecture 2}


\begin{theorem}
Let $\rho: G \to GL(n,F)$ be a representation of $G$ over $F$.
	\begin{enumerate}[label=\emph{(\Roman*)}]
	\item If $V=F^n$ is an $FG$ module and $G$ acts on $V$ by $v \cdot g = v (g\rho)$ there exists a basis $B$ of $V$ such that $g\rho = [g]_B$.
	\item The map $g \mapsto [g]_B$ is a representation for $G$ over $F$.
\end{enumerate}
\end{theorem}

\begin{proof}
%Continue
Choose the standard basis $B = [e_1, \ldots, e_n]$.

Since $V$ is an $FG$-module we have $v(gh) = (vg)h$ for all $g, h \in G$ and $v \in V$. Thus $[gh]_B = [g]_B [h]_B$ so the map is a homomorphism. We now check that $[g]_B$ is invertable for all $g \in G$. We know $v \cdot 1_G = (v g) g^{-1}$ so $I_n = [g]_B [g^{-1}]_B$ and thus $[g]_B$ has an inverse.
\end{proof}

\begin{example}
	Recall the representation of $G=D_4$ from a previous example. Define an $FG$-module $V=F^2$ with the action defined by taking $vg$ to $v(g\rho)$.
\begin{align*}
	v_1 &= (1,0), \quad v_1 a = v_2, \quad v_1 b = v_1,\\
	v_2 &= (0,1), \quad v_1 a = -v_1, \quad v_1 b = -v_2.
\end{align*}	
In this basis we recover our representation
\[
	a \mapsto [a]_B = \begin{bmatrix*}[r] 0 & 1\\ -1 & 0 \end{bmatrix*}, \quad
	b \mapsto [b]_B = \begin{bmatrix} 1 & 0\\ 0 & -1 \end{bmatrix}.
\]
\end{example}

%Similar exercise but for Q8.

We now provide an equivalent basis-dependent definition for an $FG$-module.
\begin{lemma}
	Let $V$ a vector space over the field $F = \R,\C$ with basis $B = [v_1, \ldots, v_n]$, and let $G$ be a group. If a multiplication $v \cdot g$ for $v \in B$, and $g \in G$ is defined such that:
\begin{enumerate}[label=(\roman*)]
    \item $v \cdot g \in V$;
    \item $v \cdot (gh) = (v \cdot g) \cdot h$;
    \item $v \cdot 1_G = v$;
    \item $\left( \sum_{i=1}^n \lambda_i v_i \right) \cdot g = \sum_{i=1}^n \lambda_i (v_i \cdot g)$ for all $\lambda_i \in F$;
\end{enumerate}
then $V$ is an $FG$-module.
\end{lemma}

%proof?

\begin{definition}
The trivial module of a group over $F$ is a one dimensional vector space $V$ over $F$ such that $v g = v$ for all $v \in V$ and $g \in G$.
\end{definition}

\begin{definition}
An $FG$-module is faithful if $1_G$ is the only $g \in G$ such that $v g = v$ for all $v \in V$.
\end{definition}

%Representations are equivelent for all choices of basis

\begin{theorem}
Let $V$ be an $FG$-module with basis $B$ and $\rho$ a representation of group $G$ over $F$ defined by taking $g \mapsto [g]_B$.
\begin{enumerate}[label=(\roman*)]
	\item If $B^\prime$ is another basis of $V$ then the map $g \mapsto [g]_{B^\prime}$ is a representation of $G$ equivalent to $\rho$.
	\item If representation $\sigma$ is equivalent to $\rho$ then there exists basis $B^{\prime\prime}$ such that $\sigma(g) = [g]_{B^{\prime\prime}}$ for all $g \in G$.
\end{enumerate}
\end{theorem}

\begin{proof}
Taking $T$ to be the change of basis matrices, the two representations are equivalent.
\end{proof}

\begin{example}
	Let $G = C_3 = \langle a \mid a^3 = 1 \rangle$ and representation $\rho: G \to GL(n,F)$ defined by
\[
	a \mapsto \begin{bmatrix} 0 & 1 \\ -1 & -1 \end{bmatrix}.
\]
	We attempt to construct an $FG$-module with group action described by $\rho$. Take $V = F^2$ with basis $B = [v_1, v_2]$. Define the action of $G$ on $V$ by
\[
	v_1 a = v_2, \quad v_2 a = -v_1 - v_2.
\]
	Let us now choose alternate basis $B^\prime = [u_1 = v_1, u_2 = v_1 + v_2]$. The action of $G$ on this basis is described by
\[
	u_1 a = -u_1 + u_2, \quad u_2 a = -u_1.
\]
This gives us a representation
\[
	a \mapsto [a]_{B^\prime} = \begin{bmatrix} -1 & 1\\ -1 & 0 \end{bmatrix}.
\]
To verify this construction we write our change of basis matrix as
\[
	T = \begin{bmatrix} 1 & 0\\ 1 & 1 \end{bmatrix}
\]
and verify that
\[
	\begin{bmatrix} 1 & 0\\ 1 & 1 \end{bmatrix}
	\begin{bmatrix} 0 & 1\\ -1 & -1 \end{bmatrix}
	\begin{bmatrix} 1 & 0\\ 1 & 1 \end{bmatrix}^{-1} = 
	\begin{bmatrix} -1 & 1\\ -1 & 0 \end{bmatrix}.
\]
\end{example}

\begin{definition}
	The permutation module of a group $G \leq S_n$ is an $n$-dimensional vector space $V$ with basis $B = [v_1, \ldots, v_n]$ and action by $G$ defined by
\[
	v_i g = v_{ig}
\]
for all $g \in G$ where $ig$ is the image of $i$ under $g \in S_n$.
\end{definition}

%Show that the permutation module is an FG-module.

It follows from Caley's theorem that every group has a faithful $FG$-module.

\begin{example}
Take $G = S_4$ and pick $g = (1\;2)$ and $h = (1\;2\;3\;4)$. We have representations
\[
    [g]_B = \begin{bmatrix} 0&1&0&0\\ 1&0&0&0\\ 0&0&1&0\\ 0&0&0&1 \end{bmatrix}, \quad
    [h]_B = \begin{bmatrix} 0&1&0&0\\ 0&0&1&0\\ 0&0&0&1\\ 1&0&0&0 \end{bmatrix}.
\]
\end{example}


\subsection{Module Reducibility}



\begin{definition}
Let $V$ be an $FG$-module. We call $W$ a submodule of $V$ if $W$ is a vector subspace of $V$ and $W$ is closed under the action of $G$. We then write $W<V$.
\end{definition}

\begin{example}
Let $G=C_3=\langle (1\;2\;3)\rangle$ and $V$ the permutation module of $G$ with basis $B=[v_1,v_2,v_3]$. The subspace $W=\langle v_1 + v_2 + v_3 \rangle$ is a submodule but the subspace $U=\langle v_1 + v_2 \rangle$ is not.

For example, consider the action of $g = (1\;2\;3)$ on $v_1 + v_2 \in U$. 
\[
	(v_1 + v_2)g = v_{1g} + v_{2g} = v_2 + v_3 \not\in U
\]
whereas $G$ acts on $W$ trivially.
\end{example}
















% ---------------------------------------------------------------------------------
% LECTURE 3
% ---------------------------------------------------------------------------------



\section{Lecture 3}

\subsection{Module and Representation Reducibility}

For any module, it is clear that we have two trivial submodules: $0<V$ and $V<V$. Where $0 = \{0\}\subset V$.

\begin{definition}
	Let $V$ be an FG-module. We say that $V$ is irreducible if the only submodules of $V$ are $V$ and $0$. Otherwise $V$ is reducible
\end{definition}

In 2.11 we showed that the permutation module of $C_3$ is reducible.

\begin{definition}
	Let $\rho:G\rightarrow GL(n,F)$ be a representation. We say that $\rho$ is irreducible if the corresponding $FG$-module (as constructed in 2.1) is irreducible. Otherwise $\rho$ is reducible.
\end{definition}

If an $FG$-module, $V$  is reducible, that is, $0<W<V$, $0\neq W\neq V$. Let $B_W$ be a basis for $W$. If we extend $B_W$ to $B$ a basis of $V$, then we get the following representation of $G$:

\begin{equation}\label{eq:3.1}
	g\mapsto [g]_B = \begin{bmatrix}
		X_g & 0\\Y_g & Z_g
	\end{bmatrix}
\end{equation}
Where the matrices $X_g,Y_g$ and $Z_g$ are some block matrices and $0$ is a block of zeros and $X_g$ has the dimensions $m\times m$ and, in this case, $\operatorname{dim}(W)=m$.
	
\begin{proposition}
	A representation $\rho:G\rightarrow GL(n,F)$ is reducible if and only if with respect to some basis, $B$, of $F^n$, $[g]_B$ has the form \ref{eq:3.1} for some $0<m<\operatorname{dim}(V)$ for all $g\in G$. Then the maps $g\mapsto X_g$ and $g\mapsto Z_g$ are both representations of $G$.
\end{proposition}

\begin{proof}
	Suppose we have a presentation, $\rho:G\rightarrow GL(n,F)$ and a basis $B$ of $V = F^n$ such that $[g]_B$ has the form \ref{eq:3.1} for every $g\in G$. Then consider the subspace $0\subset W\subset V$ spanned by the first $m$ elements of $B$. It is clear that $v[g]_B\in W$ for all $v\in W$. Therefore the module induced by $\rho$ is reducible, so $\rho$ is reducible. Now, if we have a reducible representation, then the argument above this proposition shows that with respect to any basis extending $B_W$, the matrices $[g]_B$ have the required form.\\
	Now, using elementary block matrix multiplication, we get the following for $g,h\in G$: 
	\[\rho(g)\rho(h) = [g]_B[h]_B = \begin{bmatrix}
	X_gX_h & 0\\
	Y_gX_h+Z_gY_h & Z_gZ_h
	\end{bmatrix} = [gh]_B=\rho(gh)\]
	Therefore $X_{gh} = X_gX_h$ and $Z_{gh} = Z_gZ_h$, so the maps $g\mapsto X_g$ and $g\mapsto Z_g$ are both representations of $G$.
\end{proof}

\begin{problem}
	Prove that the example representation of $D_8$ of degree 2 over $\R$ or $\C$ is irreducible.
\end{problem}

\subsection{Group Algebras}

Recall that an algebra over a field $F$ is a  vector space over $F$ equipped with a bilinear product $A\times A\rightarrow A$ that is compatible with scalar multiplication.

\begin{definition}
	The group algebra over a finite group $G$ over a field $F$ is an algebra\footnote{See Lemma\ref{le:3.6} } of dimension $n = |G|$ over $F=\R$ or $\C$ called $FG$, with basis $B=G = \{g_1,\dots g_n\}$. Where the algebra structure is given by the following for two arbitrary elements of $FG$, $u = \sum_{g\in G}\lambda_gg$, $v = \sum_{g\in G}\mu_g$, $\lambda_g,\mu_g\in F$ and $\nu\in F$:
	\begin{enumerate}[label=(\roman*)]
		\item $u+v = \sum_{i=1}^n(\lambda_i+\mu_i)g_i$
		\item $\nu \cdot u = \sum_{i=1}^{n}(\nu\lambda_i)g_i$
		\item $u\cdot v = \sum_{(h,g)\in G\times G}\lambda_g\mu_h(gh)$
	\end{enumerate}
\end{definition}

This is clearly a vector space.

\begin{example}
	Consider $G = C_3 = \{e,a,a^2\} = \langle a|a^3=e\rangle$ and $F=\R$ or $\C$. Then if we let $u = e-a+2a^2$, $v=\frac{1}{2}e+5a$, then:
	\[u+v = \frac{3}{2}e+4a+2a^2,\quad \frac{1}{3}u = \frac{1}{3}e-\frac{1}{3}a+\frac{2}{3}a^2,\quad uv = \frac{21}{2}e+\frac{9}{2}a-4a^2\]
\end{example}
\begin{lemma}\label{le:3.6}
	Given a group algebra $FG$, $r,s,t\in FG$, $\lambda\in F$:
	\begin{enumerate}[label=\emph{(\Roman*)}]
		\item $rs\in FG$
		\item $(rs)t = r(st)$
		\item $1_Gr=r1_G=r$
		\item $(\lambda r)s=\lambda(rs)$
		\item $(r+s)t=rt+st$
		\item $r(s+t)=rs+rt$
		\item $r0=0r=0$
	\end{enumerate}
That is, $FG$ is an associative algebra with unit
\end{lemma}
\begin{proof}
	1,3 and 7 are clear from the definition of $FG$, 4,5 and 6 follow from the distributive and associative laws of $F$ and 2 follows from associativity in $G$.
\end{proof}
\subsection{The Regular FG-module, FG}
\begin{problem}
	$V=FG$ is an $FG$-module with the group action defined by $v\cdot g=vg$ for $v\in FG$, $g\in G\subset FG$.
\end{problem}
\begin{definition}
	For a finite group $G$ and $F=\R$ or $\C$, the regular $FG$-module is $FG$. The associated module, $g\mapsto [g]_B$ is called the regular representation.
\end{definition}
\begin{lemma}
	$FG$ is a faithful module for $G$ over $F$
\end{lemma}
\begin{proof}
	If $vg=v$ for all $v\in FG$, then specifically, $hg=h$ for all $h\in G$, so $g=1_G$.
\end{proof}
\begin{example}
	For $C=C_3$, over the basis $B=G$, we get:
	\[[e]_B=I_3,\quad [a]_B = \begin{bmatrix}
	0&1&0\\0&0&1\\1&0&0
	\end{bmatrix},\quad[a^2]_B=\begin{bmatrix}
	0&0&1\\1&0&0\\0&1&0
	\end{bmatrix}\]
\end{example}
Now, if we have an $FG$-module, $V$, then $FG$ acts on $V$ in the following way:
\[v\cdot r = v\cdot \left(\sum_{g\in G}\mu_g g\right) = \sum_{g\in G}\mu_g(v\cdot g)\]
\begin{lemma}
	For $u,v\in V$, $\lambda\in F$, $r,s\in FG$:
	\begin{enumerate}[label=\emph{(\Roman*)}]
		\item $vr\in FG$
		\item $(vr)s = v(rs)$
		\item $v1=v$
		\item $(\lambda v)r=\lambda(vr)=v(\lambda r)$
		\item $(r+s)v=rv+sv$
		\item $r(u+v)=ru+rv$
		\item $r0=v0=0$
	\end{enumerate}
\end{lemma}
\begin{proof}
	1,3 and the first part of 7 follow from $V$ being an $FG$-module, the second equality of $7$ follows from scalar multiplication by 0 in $V$. The following calculation:
	\begin{align*}
	(\lambda v)r &= \sum_{g\in G}\mu_g((\lambda v)g)\\
	&=\sum_{g\in G}\mu_g(\lambda (vg))\\
	&=\sum_{g\in G}(\lambda \mu_g)(vg)=v(\lambda r)\\
	&=\lambda\sum_{g\in G}\mu_g (vg)\\
	&=\lambda(vg)\\
	\end{align*}proves 4. 6 follows from the linearity of the action of $G$ on $V$. 5 follows from distributivity of scalar multiplication in $V$. Finally, to prove 2:
	\begin{align*}
	v(rs)&=\sum_{(g,h)\in G\times G}(\mu_g\lambda_h(v(gh)))\\
	&= \sum_{h\in G}\lambda_h\sum_{g\in G}\mu_g(gv)h\\
	&= \sum_{h\in G}\lambda_h\left(\sum_{g\in G}\mu_g(gv)\right)h\\
	&=\sum_{h\in G}\lambda_h(vr)h=(vr)s
	\end{align*}
\end{proof}

















% -------------------------------------------------------------------
% LECTURE 4
% -------------------------------------------------------------------
\section{Lecture 4}

\subsection{Homomorphisms} 

\begin{definition}
	Let $V$ and $W$ be $FG$-modules. A \textit{homomorphism} of $FG$-modules is a map $\sigma: V \rightarrow W$ which is a linear transformation and also satisfies $(vg)\sigma = (v\sigma)g$ for all $g\in G, v\in V$. The \textit{kernel} and \textit{image} are defined in the obvious way
\end{definition}
Equivalently, it is a homomorphism of modules over the ring $FG$. Indeed:
\begin{problem}
	Suppose $r\in FG$ is an element of the group algebra. Prove that $(vr)\sigma = (v\sigma)r$.
\end{problem}



\begin{lemma}
	Let $\sigma: V \rightarrow W$ be a homomorphism of $FG$-algebras. Then the kernel and image of $\sigma$ are submodules
\end{lemma}
\begin{proof}
	This is a matter of simple checking, which will be left to the reader.
\end{proof}

\begin{example}
	Take $\sigma: V \rightarrow V$ to be $v \mapsto \lambda v$ for some $\lambda \in F^*$. Then $\ker \sigma = 0, \im \sigma = V$. 
\end{example}
\begin{example}
	Let $G = S_n$ and $V = \langle v_1,..., v_n \rangle$ be the permutation module for $G$ over $F$, and let $W = \langle w \rangle$ be the trivial module. Now define $\sigma: V \rightarrow W$ by
\[
	\sum \lambda_i v_i \mapsto \sum \lambda_i w
\]
Then $\ker \sigma = \{\sum \lambda_i v_i \mid \sum \lambda_i = 0\}$ and $\im \sigma = W$. 
\end{example}

\begin{definition}
	A homomorphism of $FG$-modules is an \textit{isomorphism} if it is bijective
\end{definition}
\begin{remark}
	In class we originally said "if the homomorphism has trivial kernel". However, this is definitely not correct because inclusions are always homomorphisms, but obviously not isomorphisms. 
\end{remark}

\begin{lemma}
	The inverse of an isomorphism is an isomorphism
\end{lemma}
\begin{proof}
	Once again, this is just an exercise in checking. The details will be left for the reader.
\end{proof}
Some rather obvious invariants of $FG$-modules (under isomorphism) are dimension and irreducibility.
\begin{lemma}
	$V$ and $W$ are isomorphic if and only if there exists bases $\mathcal{B}_1$ of $V$ and $\mathcal{B}_2$ of $W$ such that
\[
	[g]_{\mathcal{B}_1} = [g]_{\mathcal{B}_2}
\]
for all $g$.
\end{lemma}
\begin{proof}
	Suppose firstly that $V$ and $W$ are isomorphic, and let $\sigma: V \rightarrow W$ be one such isomorphism. Let $\mathcal{B}_1 = \{v_1,...,v_n\}$ be a basis for $V$. In particular, it is linearly independent, and it is easy to see that $\mathcal{B}_2 = \{v_1\sigma,...,v_n\sigma\}$ is also linearly independent. Since $V$ and $W$ are isomorphic, they have the same dimension, and thus $\mathcal{B}_2$ is a basis for $W$. Since $(vg)\sigma = (v\sigma)g$ for all $g$ and $v$, the action of $g$ on the basis vectors of both bases are the same, and thus we conclude $[g]_{\mathcal{B}_1} = [g]_{\mathcal{B}_2}$.
	\\\\
	Conversely, suppose that the latter hypothesis is satisfied. Let $\{v_1,...,v_n\}$ be a basis for $V$ and $\{w_1,...,w_n\}$ be a basis for $W$. We define a bijective linear map $\sigma: V \rightarrow W$ such that $v_i \sigma = w_i$ for each $i$. Now observe that for each $i$, we have  $v_ig = \lambda_1 v_1 + ... + \lambda_n v_n$ and $w_ig = \lambda_1 w_1 +...+ \lambda_n w_n$, where $(\lambda_1,...,\lambda_n)$ is the $i$-th row of $[g]$. This means that
\[
	(v_ig)\sigma = (\lambda_1 v_1 + ... + \lambda_n v_n)\sigma = \lambda_1 v_1\sigma + ... + \lambda_n v_n\sigma = \lambda_1 w_1 +...+ \lambda_n w_n = w_ig = (v_i\sigma)g
\]
and thus $\sigma$ is a homomorphism of $FG$-modules. Since it is bijective, it is an isomorphism.
\end{proof}
\begin{theorem}
	Let $V$ be an $FG$-module with basis $\mcl{B}_1$ and $W$ an $FG$-module with basis $\mcl{B}_2$. Then $W \cong V$ if and only if $g\mapsto [g]_{\mcl{B}_1}$ and $g\mapsto [g]_{\mcl{B}_2}$ are equivalent.
\end{theorem}
\begin{proof}
	This follows from the previous Lemma and the fact that two matrices are conjugate ($A$ and $B$ are conjugate if $A = P^{-1}B P$ for some $P$) if and only if the linear transformations they define differ by a change of basis (that is they define the same transformation but with respect to different bases)
\end{proof}

\begin{example}
	Let $G = C_3 = \{e, a, a^2\}$. Let $V$ be the regular representation, that is the natural representation induced by the module $FG = \langle e, a, a^2 \rangle$. Write $B:= \{e, a, a^2\}$ as a basis for $FG$. Then
\[
	[a]_B = \left(\begin{matrix}
	0 & 1 & 0 \\ 0 & 0 & 1 \\ 1 & 0 & 0
	\end{matrix}\right)
\]
Now let $W$ be the permutation module where $a = (1, 2, 3)$ and $C_3$ is considered a subgroup of $S_3$. Write $B' = \{v_1, v_2, v_3\}$ for the basis of $W$. Then
\[
	[a]_{B'} = \left(\begin{matrix}
	0 & 1 & 0 \\ 0 & 0 & 1 \\ 1 & 0 & 0
	\end{matrix}\right)
\]
	Note that these two modules are isomorphic. 
\end{example}
\begin{example}
	Let $G = D_4 = \langle a, b \mid a^4 = b^2 = 1, a^b = a^{-1}\rangle$. Now we can act on either $F^4$ or $F^8$. On $F^4$, we have the representation described in \thref{d4}. On $W$, we have the regular representation. Clearly are not isomorphic.
\end{example}

\subsection{Sums}

We now consider how modules behave with respect to direct sums. Let $V$ be an $FG$-module and suppose $V = U \oplus W$, where $U$ and $W$ are submodules. Let $\mcl{B}_1 =\{u_1,...,u_n\}$ be a basis for $U$ and $\mcl{B}_2 =\{w_1,...,w_m\}$ one for $W$, so that $\mcl{B} = \mcl{B}_1 \cup \mcl{B_2}$ is a basis for $V$. Then 
\[
    [g]_{\mcl{B}} = \begin{pmatrix*}
        [g]_{\mcl B_1} & 0 \\0 & [g]_{\mcl{B}_2}
    \end{pmatrix*}
\]
%(I have no idea why, but the moment I try use square brackets around that $g$ it fucks up - Oliver). 

\begin{lemma}
	Let $V$ be an $FG$-module such that we have the decomposition
\[
	V = \bigoplus_{i = 1}^n U_i
\]
Define the projection map $\pi_i: u_1 + u_2 +...+ u_n \mapsto u_i$. Then 
	\begin{enumerate}[label=\emph{(\Roman*)}]
		\item $\pi_i$ is a homomorphism
		\item $\pi_i \circ \pi_i = \pi_i$
	\end{enumerate}
\end{lemma}
\begin{proof}
	Trivial
\end{proof}

\begin{lemma}
	Suppose we have a finite decomposition
\[
	V = \sum U_i
\]
where the $U_i$ are irreducible. Then $V$ is the direct sum of the subset of the $U_i$. 
\end{lemma}
\begin{proof}
	This follows from the fact that the intersection of two distinct irreducible modules is trivial (again, simple checking).
\end{proof}

We will now present an important result
\begin{theorem} [Maschke's Theorem]
	Let $G$ be a finite group, $F$ a field of characteristic 0, $V$ an $FG$-module and $U$ a submodule. Then there exists some $W$ such that $V = U \oplus W$.
\end{theorem}
\begin{proof}
	We first choose some $W_1$ such that $V = U \oplus W_1$ as vector spaces. Note that each $v\in V$ can be uniquely decomposed as $v = u + w$ , where $u\in U, w\in W_1$. Now define the canonical projection $\sigma: V \rightarrow U$ where $v \mapsto u$. Clearly $\ker \sigma = W_1$ and $\im \sigma = U$. However, we note that $\sigma$ is NOT necessarily a homomorphism of $FG$-modules. We modify it as follows: Define $\varphi: V \rightarrow V$ by
\[
	v\mapsto \frac{1}{{|G|}} \sum_{g\in G} vg\sigma g^{-1}
\]
We claim that $\varphi$ IS a homomorphism. Indeed, suppose $x\in G, v\in V$. Then 
	\begin{align*}
		(xv)\varphi &= \frac{1}{{|G|}} \sum_{g\in G} (vx)g\sigma g^{-1}\\
		&= \frac{1}{{|G|}} \sum_{h\in G} vh\sigma h^{-1}x \\
		&= \left(\frac{1}{{|G|}} \sum_{h\in G} vh\sigma h^{-1}\right)x = (v\varphi)x
	\end{align*}
	where the equality
\[
	\frac{1}{{|G|}} \sum_{g\in G} (vx)g\sigma g^{-1}= \frac{1}{{|G|}} \sum_{h\in G} vh\sigma h^{-1}x
\]
follows from the change of variables $h = xg$. Clearly $\varphi$ maps into $U$, and we now check it is a projection. Indeed, supposing $u\in U$ we have 
	\begin{align*}
		(u)\varphi &= \frac{1}{{|G|}} \sum_{g\in G} ug\sigma g^{-1} \\
		&= \frac{1}{{|G|}} \sum_{g\in G} u\sigma gg^{-1} \\
		&= \frac{1}{{|G|}} \sum_{g\in G} u\\
		&= u
	\end{align*}
	as desired. 
	\\\\
	Now clearly $U = \im \varphi$ and we define $W:= \ker \varphi$. Then for each $v\in V$, write $u:= v\varphi \in U$ and $w:= v - u\in W$ so that $v = u + w$. It only remains to check that this is unique. To see this, suppose $u' + w' = v = u + w$. Then
\[
	u' = \varphi(u') = \varphi(v) = \varphi(u) = u
\]
which implies the result.
\end{proof}

\end{document}
