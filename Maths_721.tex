\documentclass[11pt, notitlepage]{article}

\usepackage[english]{babel}
\usepackage[utf8x]{inputenc}
\usepackage{amsmath}
\usepackage{amssymb}
\usepackage{mathtools}
\usepackage{amssymb}
\usepackage{amsfonts}
\usepackage{mathdots}
\usepackage{multicol}
\usepackage{array}
\usepackage{cool}
\usepackage{parskip}
\usepackage{tikz}
\usetikzlibrary{automata, arrows.meta, chains}
\usepackage[a4paper]{geometry}
\usepackage{tensor}

\usepackage{color}
\usepackage{siunitx}
\usepackage{hyperref}
\usepackage{amsthm}
\usepackage{enumitem}
\usepackage{tikz-cd}
\usepackage{mathrsfs}
\usepackage{natbib}
\usepackage{fancyhdr}
\usepackage[nottoc]{tocbibind}


\pagestyle{fancy}
\fancyhf{}
\fancyhead[R]{\textit{\rightmark}}
\fancyfoot[C]{Page \thepage}

\usepackage{microtype}
\DisableLigatures[f]{encoding = *, family = *}


\newcommand\mapsfrom{\mathrel{\reflectbox{\ensuremath{\mapsto}}}}



\renewcommand{\footrulewidth}{0.5pt}

\bibliographystyle{alpha}
\addto\captionsenglish{\renewcommand{\bibname}{References}}

%\usepackage[backref=page,pagebackref=true,linkcolor = blue,citecolor = red]{hyperref}
%\usepackage[backref=page]{backref}


\usepackage{graphicx}
\DeclareGraphicsExtensions{.pdf,.png,.jpg}

\numberwithin{equation}{section}

\theoremstyle{plain}
\newtheorem{theorem}{Theorem}[section]
\newtheorem{corollary}{Corollary}[theorem]
\newtheorem{lemma}[theorem]{Lemma}
\newtheorem{proposition}[theorem]{Proposition}


\theoremstyle{definition}
\newtheorem{definition}[theorem]{Definition}
\newtheorem{problem}{Problem}
\newtheorem{remarkx}{Remark}
\newenvironment{remark}
	{\pushQED{\qed}\renewcommand{\qedsymbol}{$\blacklozenge$}\remarkx}
	{\popQED\endremarkx}
	
\newtheorem{examplex}[theorem]{Example}
\newenvironment{example}
	{\pushQED{\qed}\renewcommand{\qedsymbol}{$\blacktriangleleft$}\examplex}
	{\popQED\endexamplex}


% Syntax for below is \cv{a_1,a_2,...,a_n}.
% Creates a square bracketed column vector.
\makeatletter
\newcommand{\cv}[2][r]{%
  \gdef\@VORNE{1}
  \left[\hskip-\arraycolsep%
    \begin{array}{#1}\vekSp@lten{#2}\end{array}%
    \hskip-\arraycolsep\right]}

\def\vekSp@lten#1{\xvekSp@lten#1,vekL@stLine,}
\def\vekL@stLine{vekL@stLine}
\def\xvekSp@lten#1,{\def\temp{#1}%
  \ifx\temp\vekL@stLine
  \else
    \ifnum\@VORNE=1\gdef\@VORNE{0}
    \else\@arraycr\fi%
    #1%
    \expandafter\xvekSp@lten
  \fi}
\makeatother

\makeatletter
\renewcommand*\env@matrix[1][\arraystretch]{%
  \edef\arraystretch{#1}%
  \hskip -\arraycolsep
  \let\@ifnextchar\new@ifnextchar
  \array{*\c@MaxMatrixCols c}}
\makeatother

\newcommand{\V}{\mathcal{V}}
\newcommand{\calL}{\mathcal{L}}
\newcommand{\A}{\mathbb{A}}
\newcommand{\R}{\mathbb{R}}
\newcommand{\C}{\mathbb{C}}
\newcommand{\Z}{\mathbb{Z}}
\newcommand{\Q}{\mathbb{Q}}
\newcommand{\N}{\mathbb{N}}
\newcommand{\F}{\mathbb{F}}
\newcommand{\T}{\mathbb{T}}
\renewcommand{\E}{\mathbb{E}}
\newcommand{\K}{\mathbb{K}}
\newcommand{\B}{\mathbb{B}}
\newcommand{\sph}{\mathbb{S}}
\newcommand{\Halfspace}{\mathbb{H}}
\renewcommand{\P}{\mathbb{P}}
\newcommand{\inner}[2]{\left\langle #1,#2 \right\rangle}
\newcommand{\tbasis}[1]{\frac{\partial}{\partial #1}}
\newcommand{\extp}[1]{\bigwedge\nolimits^{#1}}



\newcommand{\abs}[1]{\left\lvert#1\right\rvert}
\newcommand{\norm}[1]{\left\lVert#1\right\rVert}
\DeclareMathOperator{\proj}{proj}
\DeclareMathOperator{\cis}{cis}
\let\Arg\relax
\DeclareMathOperator{\Arg}{Arg}
\DeclareMathOperator{\col}{col}
\DeclareMathOperator{\rank}{rk}
\DeclareMathOperator{\row}{row}
\DeclareMathOperator{\nul}{null}
\DeclareMathOperator{\spn}{span}
\DeclareMathOperator{\Mat}{Mat}
\let\hom\relax
\DeclareMathOperator{\hom}{Hom}
\DeclareMathOperator{\epi}{epi}
\DeclareMathOperator{\Sym}{Sym}
\DeclareMathOperator{\GL}{GL}
\DeclareMathOperator{\SL}{SL}
\DeclareMathOperator{\sgn}{sgn}
\DeclareMathOperator{\Aff}{Aff}
\DeclareMathOperator{\lcm}{lcm}
\DeclareMathOperator{\Isom}{\textbf{Isom}}
\DeclareMathOperator{\gen}{gen}
\DeclareMathOperator{\im}{im}
\DeclareMathOperator{\Aut}{Aut}
\DeclareMathOperator{\Inn}{Inn}
\DeclareMathOperator{\Sub}{\textbf{Sub}}
\DeclareMathOperator{\ind}{ind}
\DeclareMathOperator{\exterior}{Ext}
\DeclareMathOperator{\interior}{Int}
\DeclareMathOperator{\identity}{Id}
\DeclareMathOperator{\orthogonal}{O}
\DeclareMathOperator{\supp}{supp}
\DeclareMathOperator{\real}{Re}
\DeclareMathOperator{\imagine}{Im}
\DeclareMathOperator{\Ind}{Ind}
\DeclareMathOperator{\SRW}{SRW}
\DeclareMathOperator{\Markov}{Markov}
\DeclareMathOperator{\tr}{tr}
\DeclareMathOperator{\Poisson}{Poisson}
\DeclareMathOperator{\Geometric}{Geometric}
\DeclareMathOperator{\Exponential}{Exp}
\DeclareMathOperator{\cl}{cl}
\DeclareMathOperator{\Diff}{Diff}
\DeclareMathOperator{\ad}{ad}
\DeclareMathOperator{\Ad}{Ad}
\DeclareMathOperator{\unitary}{U}
\DeclareMathOperator{\End}{End}
\DeclareMathOperator{\inprod}{\lrcorner}
\DeclareMathOperator{\rot}{rot}
\DeclareMathOperator{\grad}{grad}
\DeclareMathOperator{\first}{I}
\DeclareMathOperator{\second}{II}
\DeclareMathOperator{\third}{III}
\DeclareMathOperator{\projection}{pr}
\DeclareMathOperator{\coker}{coker}
\DeclareMathOperator{\vol}{vol}
\DeclareMathOperator{\SO}{SO}


\def\*#1{\mathbf{#1}}
\newcommand{\veps}{\varepsilon}
\newcommand{\vphi}{\varphi}
\newcommand{\normsub}{\unlhd}
\newcommand{\snormsub}{\lhd}
\title{Maths 721 Notes}
\date{2020}

\begin{document}

\maketitle
\tableofcontents
\vspace{5mm}


% -------------------------------------------------------------------
% LECTURE 1
% -------------------------------------------------------------------

\pagebreak

\section{Lecture 1}


In the first half of this course we will cover three main topics:
\begin{itemize}
    \item representations;
    \item modules;
    \item characters.
\end{itemize}
We will further see that representations and modules are essentially the same, and that modules and characters are essentially the same; and hence all three are essentially the same.

From now on $G$ is a group.



\subsection{Representations}


\begin{definition}
A \textbf{representation} of a group $G$ over a field $F$ is a group homomorphism from $G$ to $\GL(n,F)$, where $n$ is the \textbf{degree} of the representation. 
\end{definition}

Explicitly, a representation is a function $\rho : G \to \GL(n,F)$ such that for all $g,h \in G$;
\begin{enumerate}[label=(\roman*)]
    \item $(gh)\rho = (g\rho)(h\rho)$,
    \item $1_G\rho = I_n$,
    \item $g^{-1}\rho = (g\rho)^{-1}$.
\end{enumerate}
Note the use of the (incredibly shit) postfix function notation.

\begin{example}
Take $D_4$, the Dihedral group of order 8. It has the following group presentations
\begin{align*}
    D_4 &= \langle a,b \mid a^4 = 1, b^2 = 1, a^b = a^{-1} \rangle\\
    &\cong \langle (1 \; 2 \; 3 \; 4), (1 \; 4)(2 \; 3) \rangle,
\end{align*}
where $a^b = bab^{-1}$ is conjugation of $a$ by $b$. By defining the matrix subgroup
\[
    H = \left\langle A = \begin{bmatrix*}[r]
        0 & 1\\
        -1 & 0
    \end{bmatrix*}, B = \begin{bmatrix*}[r]
        1 & 0\\
        0 & -1
    \end{bmatrix*}\right\rangle
\]
and defining $\rho : D_4 \to \GL(n,F)$ where $F = \R, \C$, by $a \mapsto A$ and $b \mapsto B$, and
$a^ib^j \mapsto A^iB^j$ for $0 \le i \le 3$, and $0 \le j \le 1$. Hence we have $\rho$ is a representation of $D_4$ over $F$.
\end{example}

\begin{example}
Take $\Q_8$ the Quaternion group of order 8, which has the following group presentations
\begin{align*}
    \Q_8 &= \langle a,b \mid a^4 = 1, a^2 = b^2, a^b = a^{-1}\rangle\\
    &\cong \langle \bar a = (1 \; 6 \; 2 \; 5)(3 \; 8 \; 4 \; 7), \bar b = (1 \; 4 \; 2 \; 3)(5 \; 7 \; 6 \; 8)\rangle
\end{align*}
Define
\[
    H = \left\langle A = \begin{bmatrix*}[r]
        i & 0\\
        0 & -i
    \end{bmatrix*}, B = \begin{bmatrix*}[r]
        0 & 1\\
        -1 & 0
    \end{bmatrix*}\right\rangle \subset \GL(2,\C).
\]
Then $\rho : \Q_8 \to \GL(2,\C)$ defined by $a^kb^\ell \mapsto A^kB^\ell$ is a group representation of $\Q_8$ over $\C$ of degree 2.
\end{example}


\begin{definition}
Let $G$ be a group and define
\begin{align*}
    \rho : G &\to \GL(n,F)\\
    g\rho &= I_n
\end{align*}
for all $g \in G$. Then $\rho$ is a representation, called the \textbf{trivial representation} of arbitrary of degree.
\end{definition}

It follows from the trivial representation that any group $G$ has a representation of an arbitrary degree.

Let $\rho : G \to \GL(n,F)$ be a group homomorphism, and take $T \in \GL(n,F)$. Then
\[
    (T^{-1}AT)(T^{-1}BT) = T^{-1}(AB)T.
\]
Thus, given $\rho$ define $\sigma$ such that
\[
    g\sigma = T^{-1}(g\rho)T
\]
for all $g \in G$. As for all $g,h \in G$, one has
\begin{align*}
    (gh)\sigma &= T^{-1}\big((gh)\rho\big)T\\
    &= T^{-1}(g\rho)(h\rho)T\\
    &= T^{-1}(g\rho)TT^{-1}(h\rho)T\\
    &= (g\sigma)(h\sigma),
\end{align*}
and so $\sigma$ is a group homomorphism; and hence a representation.

\begin{definition}
Define
\[
    \rho : G \to \GL(m,F), \qquad \sigma : G \to \GL(n,F)
\]
to both be representation of $G$ over $F$. We say that \textbf{$\rho$ is equivalent to $\sigma$} if $n=m$ and there exists $T \in \GL(n,F)$ such that $g\sigma = T^{-1}(g\rho)T$.
\end{definition}

\begin{proposition}
Equivalence of representations is an equivalence relation.
\end{proposition}

\begin{proof}
Reflexivity is clear by taking $T = I_n$. For symmetry, take $T$ to be its inverse. For transitivity, if
\[
    g\sigma = T^{-1}(g\rho)T, \qquad g\rho = S^{-1}(g\eta)S,
\]
then
\[
    g\sigma = (ST)^{-1}(g\eta)(ST).
\]
\end{proof}

\begin{definition}
Define the \textbf{kernel} of the representation $\rho : G \to \GL(n,F)$ as $\ker \rho = \{g \in G \mid g\rho = I_n\}$.
\end{definition}

\begin{proposition}
The kernel of a representation of $G$ is a normal subgroup of $G$; i.e. $\ker \rho \lhd G$.
\end{proposition}

\begin{proof}
Suppose $g \in \ker \rho$ and $h \in G$ is arbitrary. Then
\[
    (hgh^{-1})\rho = (h\rho)(g\rho)(h^{-1}\rho) = (h\rho)I_n(h\rho)^{-1} = (h\rho)(h\rho)^{-1} = I_n,
\]
and so $hgh^{-1} \in \ker \rho$. As $\ker \rho$ is closed under conjugation, it is a normal subgroup of $G$.
\end{proof}

\begin{definition}
We say $\rho$ is a \textbf{faithful} representation of $G$ if $\ker \rho = \{1_G\}$. 
\end{definition}

\begin{example}
For the trivial representation $\rho : G \to \GL(n,F)$ with $g \mapsto I_n$ for all $g \in G$, we have $\ker \rho = G$. Hence the representation is not faithful.
\end{example}

\begin{lemma}
Suppose $G$ is a finite group, and $\rho$ is a representation of $G$ over $F$. Then $\rho$ is faithful if, and only if, $\im \rho \cong G$.
\end{lemma}

\begin{proof}
Immediate from the first isomorphism theorem.
\end{proof}



\subsection{$FG$-Modules}



Suppose $G$ is a group, and $F = \R,\C$. Given $\rho : G \to \GL(n,F)$, with $V = F^n$, let $v = (\lambda_1,\dots,\lambda_n) \in V$ for $\lambda_i \in F$ be a row vector. Moreover, note that $g\rho$ is an $n \times n$ matrix for all $g \in G$. Thus, we have $v \cdot (g\rho) \in V$, and satisfies the following properties:
\begin{enumerate}[label=(\roman*)]
    \item $v \cdot \big( (gh)\rho \big) = v \cdot (g\rho)(h\rho)$;
    \item $v \cdot (1_G\rho) = v$;
    \item $(\lambda v) \cdot (g\rho) = \lambda \big( v \cdot (g\rho) \big)$;
    \item $(u + v) \cdot (g\rho) = u \cdot (g\rho) + v \cdot (g\rho)$.
\end{enumerate}

We often will omitted the $\cdot$ in the operation, and write $v(a\rho)$ for $v \cdot (a\rho)$.

\begin{example}
Recall $D_4$ and its given presentation from a previous example. We have
\[
    a\rho = \begin{bmatrix*}[r]
        0 & 1\\
        -1 & 0
    \end{bmatrix*}, \qquad b\rho = \begin{bmatrix*}[r]
        1 & 0\\
        0 & -1
    \end{bmatrix*}.
\]
If $v = (\lambda_1,\lambda_2)$, then we have
\[
    v(a\rho) = (-\lambda_2,\lambda_1), \qquad v(b\rho) = (\lambda_1,-\lambda_2).
\]
\end{example}


\begin{definition}
Let $V$ a vector space over the field $F = \R,\C$, and let $G$ be a group. We say $V$ is a \textbf{$FG$-module} if a multiplication $v \cdot g$ for $v \in V$, and $g \in G$ is defined such that:
\begin{enumerate}[label=(\roman*)]
    \item $v \cdot g \in V$;
    \item $v \cdot (gh) = (v \cdot g) \cdot h$;
    \item $v \cdot 1_G = v$;
    \item $(\lambda v) \cdot g = \lambda(v \cdot g)$;
    \item $(u + v) \cdot g = u \cdot g + v \cdot g$.
\end{enumerate}
\end{definition}

This generalises the previous discussion from a matrix group to an arbitrary group.

Note that properties (i), (iv), and (v) imply that the map $v \mapsto v \cdot g$ is an endomorphism of $V$ (a linear map from $V$ to itself).

\begin{definition}
Suppose $V$ is an $FG$-module and $B$ is a basis for $V$. For $g \in G$, let $[g]_B$ denote the matrix of the endomorphism $v \mapsto v \cdot g$ of $V$ relative to the basis $B$.
\end{definition}


\section{Lecture 2}


\begin{theorem}
Let $\rho: G \to GL(n,F)$ be a representation of $G$ over $F$.
	\begin{enumerate}[label=\emph{(\Roman*)}]
	\item If $V=F^n$ is an $FG$ module and $G$ acts on $V$ by $v \cdot g = v (g\rho)$ there exists a basis $B$ of $V$ such that $g\rho = [g]_B$.
	\item The map $g \mapsto [g]_B$ is a representation for $G$ over $F$.
\end{enumerate}
\end{theorem}

\begin{proof}
%Continue
Choose the standard basis $B = [e_1, \ldots, e_n]$.

Since $V$ is an $FG$-module we have $v(gh) = (vg)h$ for all $g, h \in G$ and $v \in V$. Thus $[gh]_B = [g]_B [h]_B$ so the map is a homomorphism. We now check that $[g]_B$ is invertable for all $g \in G$. We know $v \cdot 1_G = (v g) g^{-1}$ so $I_n = [g]_B [g^{-1}]_B$ and thus $[g]_B$ has an inverse.
\end{proof}

\begin{example}
	Recall the representation of $G=D_4$ from a previous example. Define an $FG$-module $V=F^2$ with the action defined by taking $vg$ to $v(g\rho)$.
\begin{align*}
	v_1 &= (1,0), \quad v_1 a = v_2, \quad v_1 b = v_1,\\
	v_2 &= (0,1), \quad v_1 a = -v_1, \quad v_1 b = -v_2.
\end{align*}	
In this basis we recover our representation
\[
	a \mapsto [a]_B = \begin{bmatrix} 0 & 1\\ -1 & 0 \end{bmatrix}, \quad
	b \mapsto [b]_B = \begin{bmatrix} 1 & 0\\ 0 & -1 \end{bmatrix}.
\]
\end{example}

%Similar exercise but for Q8.

We now provide an equivalent basis-dependent definition for an $FG$-module.
\begin{lemma}
	Let $V$ a vector space over the field $F = \R,\C$ with basis $B = [v_1, \ldots, v_n]$, and let $G$ be a group. If a multiplication $v \cdot g$ for $v \in B$, and $g \in G$ is defined such that:
\begin{enumerate}[label=(\roman*)]
    \item $v \cdot g \in V$;
    \item $v \cdot (gh) = (v \cdot g) \cdot h$;
    \item $v \cdot 1_G = v$;
    \item $\left( \sum_{i=1}^n \lambda_i v_i \right) \cdot g = \sum_{i=1}^n \lambda_i (v_i \cdot g)$ for all $\lambda_i \in F$;
\end{enumerate}
then $V$ is an $FG$-module.
\end{lemma}

%proof?

\begin{definition}
The trivial module of a group over $F$ is a one dimensional vector space $V$ over $F$ such that $v g = v$ for all $v \in V$ and $g \in G$.
\end{definition}

\begin{definition}
An $FG$-module is faithful if $1_G$ is the only $g \in G$ such that $v g = v$ for all $v \in V$.
\end{definition}

%Representations are equivelent for all choices of basis

\begin{theorem}
Let $V$ be an $FG$-module with basis $B$ and $\rho$ a representation of group $G$ over $F$ defined by taking $g \mapsto [g]_B$.
\begin{enumerate}[label=(\roman*)]
	\item If $B^\prime$ is another basis of $V$ then the map $g \mapsto [g]_{B^\prime}$ is a representation of $G$ equivalent to $\rho$.
	\item If representation $\sigma$ is equivalent to $\rho$ then there exists basis $B^{\prime\prime}$ such that $\sigma(g) = [g]_{B^{\prime\prime}}$ for all $g \in G$.
\end{enumerate}
\end{theorem}

\begin{proof}
Taking $T$ to be the change of basis matrices, the two representations are equivalent.
\end{proof}

\begin{example}
	Let $G = C_3 = \langle a \mid a^3 = 1 \rangle$ and representation $\rho: G \to GL(n,F)$ defined by
\[
	a \mapsto \begin{bmatrix} 0 & 1 \\ -1 & -1 \end{bmatrix}.
\]
	We attempt to construct an $FG$-module with group action described by $\rho$. Take $V = F^2$ with basis $B = [v_1, v_2]$. Define the action of $G$ on $V$ by
\[
	v_1 a = v_2, \quad v_2 a = -v_1 - v_2.
\]
	Let us now choose alternate basis $B^\prime = [u_1 = v_1, u_2 = v_1 + v_2]$. The action of $G$ on this basis is described by
\[
	u_1 a = -u_1 + u_2, \quad u_2 a = -u_1.
\]
This gives us a representation
\[
	a \mapsto [a]_{B^\prime} = \begin{bmatrix} -1 & 1\\ -1 & 0 \end{bmatrix}.
\]
To verify this construction we write our change of basis matrix as
\[
	T = \begin{bmatrix} 1 & 0\\ 1 & 1 \end{bmatrix}
\]
and verify that
\[
	\begin{bmatrix} 1 & 0\\ 1 & 1 \end{bmatrix}
	\begin{bmatrix} 0 & 1\\ -1 & -1 \end{bmatrix}
	\begin{bmatrix} 1 & 0\\ 1 & 1 \end{bmatrix}^{-1} = 
	\begin{bmatrix} -1 & 1\\ -1 & 0 \end{bmatrix}.
\]
\end{example}

\begin{definition}
	The permutation module of a group $G \leq S_n$ is an $n$-dimensional vector space $V$ with basis $B = [v_1, \ldots, v_n]$ and action by $G$ defined by
\[
	v_i g = v_{ig}
\]
for all $g \in G$ where $ig$ is the image of $i$ under $g \in S_n$.
\end{definition}

%Show that the permutation module is an FG-module.

It follows from Caley's theorem that every group has a faithful $FG$-module.

\begin{example}
Take $G = S_4$ and pick $g = (1\;2)$ and $h = (1\;2\;3\;4)$. We have representations
\[
[g]_B = \begin{bmatrix} 0&1&0&0\\ 1&0&0&0\\ 0&0&1&0\\ 0&0&0&1 \end{bmatrix}, \quad
[h]_B = \begin{bmatrix} 0&1&0&0\\ 0&0&1&0\\ 0&0&0&1\\ 1&0&0&0 \end{bmatrix}.
\]
\end{example}


\subsection{Module Reducibility}



\begin{definition}
Let $V$ be an $FG$-module. We call $W$ a submodule of $V$ if $W$ is a vector subspace of $V$ and $W$ is closed under the action of $G$. We then write $W<V$.
\end{definition}

\begin{example}
Let $G=C_3=\langle (1\;2\;3)\rangle$ and $V$ the presentation module of $G$ with basis $B=[v_1,v_2,v_3]$. The subspace $W=\langle v_1 + v_2 + v_3 \rangle$ is a submodule but the subspace $U=\langle v_1 + v_2 \rangle$ is not.

For example, consider the action of $g = (1\;2\;3)$ on $v_1 + v_2 \in U$. 
\[
	(v_1 + v_2)g = v_{1g} + v_{2g} = v_2 + v_3 \not\in U
\]
whereas $G$ acts on $W$ trivially.
\end{example}
\section{Lecture 3}

Simon's first section



\end{document}
