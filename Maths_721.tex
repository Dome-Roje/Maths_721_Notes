\documentclass[11pt, notitlepage]{article}

\usepackage[english]{babel}
\usepackage[utf8x]{inputenc}
\usepackage{amsmath}
\usepackage{amssymb}
\usepackage{mathtools}
\usepackage{amssymb}
\usepackage{amsfonts}
\usepackage{mathdots}
\usepackage{multicol}
\usepackage{array}
\usepackage{cool}
\usepackage{parskip}
\usepackage{tikz}
\usetikzlibrary{automata, arrows.meta, chains}
\usepackage[a4paper]{geometry}
\usepackage{tensor}

\usepackage{color}
\usepackage{siunitx}
\usepackage{hyperref}
\usepackage{amsthm}
\usepackage{enumitem}
\usepackage{tikz-cd}
\usepackage{mathrsfs}
\usepackage{natbib}
\usepackage{fancyhdr}
\usepackage[nottoc]{tocbibind}
\usepackage{theoremref}


\pagestyle{fancy}
\fancyhf{}
\fancyhead[R]{\textit{\rightmark}}
\fancyfoot[C]{Page \thepage}

\usepackage{microtype}
\DisableLigatures[f]{encoding = *, family = *}


\newcommand\mapsfrom{\mathrel{\reflectbox{\ensuremath{\mapsto}}}}



\renewcommand{\footrulewidth}{0.5pt}

\bibliographystyle{alpha}
\addto\captionsenglish{\renewcommand{\bibname}{References}}

%\usepackage[backref=page,pagebackref=true,linkcolor = blue,citecolor = red]{hyperref}
%\usepackage[backref=page]{backref}


\usepackage{graphicx}
\DeclareGraphicsExtensions{.pdf,.png,.jpg}

\numberwithin{equation}{section}

\theoremstyle{plain}
\newtheorem{theorem}{Theorem}[section]
\newtheorem{corollary}{Corollary}[theorem]
\newtheorem{lemma}[theorem]{Lemma}
\newtheorem{proposition}[theorem]{Proposition}


\theoremstyle{definition}
\newtheorem{definition}[theorem]{Definition}
\newtheorem{problem}{Problem}
\newtheorem{remarkx}{Remark}
\newenvironment{remark}
	{\pushQED{\qed}\renewcommand{\qedsymbol}{$\blacklozenge$}\remarkx}
	{\popQED\endremarkx}
	
\newtheorem{examplex}[theorem]{Example}
\newenvironment{example}
	{\pushQED{\qed}\renewcommand{\qedsymbol}{$\blacktriangleleft$}\examplex}
	{\popQED\endexamplex}


% Syntax for below is \cv{a_1,a_2,...,a_n}.
% Creates a square bracketed column vector.
\makeatletter
\newcommand{\cv}[2][r]{%
  \gdef\@VORNE{1}
  \left[\hskip-\arraycolsep%
    \begin{array}{#1}\vekSp@lten{#2}\end{array}%
    \hskip-\arraycolsep\right]}

\def\vekSp@lten#1{\xvekSp@lten#1,vekL@stLine,}
\def\vekL@stLine{vekL@stLine}
\def\xvekSp@lten#1,{\def\temp{#1}%
  \ifx\temp\vekL@stLine
  \else
    \ifnum\@VORNE=1\gdef\@VORNE{0}
    \else\@arraycr\fi%
    #1%
    \expandafter\xvekSp@lten
  \fi}
\makeatother

\makeatletter
\renewcommand*\env@matrix[1][\arraystretch]{%
  \edef\arraystretch{#1}%
  \hskip -\arraycolsep
  \let\@ifnextchar\new@ifnextchar
  \array{*\c@MaxMatrixCols c}}
\makeatother

\newcommand{\mcl}{\mathcal}

\newcommand{\V}{\mathcal{V}}
\newcommand{\calL}{\mathcal{L}}
\newcommand{\A}{\mathbb{A}}
\newcommand{\R}{\mathbb{R}}
\newcommand{\C}{\mathbb{C}}
\newcommand{\CG}{{\mathbb{C}G}}
\newcommand{\Z}{\mathbb{Z}}
\newcommand{\Q}{\mathbb{Q}}
\newcommand{\N}{\mathbb{N}}
\newcommand{\F}{\mathbb{F}}
\newcommand{\T}{\mathbb{T}}
\renewcommand{\E}{\mathbb{E}}
\newcommand{\K}{\mathbb{K}}
\newcommand{\B}{\mathbb{B}}
\newcommand{\sph}{\mathbb{S}}
\newcommand{\Halfspace}{\mathbb{H}}
\renewcommand{\P}{\mathbb{P}}
\newcommand{\inner}[2]{\left\langle #1,#2 \right\rangle}
\newcommand{\tbasis}[1]{\frac{\partial}{\partial #1}}
\newcommand{\extp}[1]{\bigwedge\nolimits^{#1}}



\newcommand{\abs}[1]{\left\lvert#1\right\rvert}
\newcommand{\norm}[1]{\left\lVert#1\right\rVert}
\DeclareMathOperator{\proj}{proj}
\DeclareMathOperator{\cis}{cis}
\let\Arg\relax
\DeclareMathOperator{\Arg}{Arg}
\DeclareMathOperator{\col}{col}
\DeclareMathOperator{\rank}{rk}
\DeclareMathOperator{\row}{row}
\DeclareMathOperator{\nul}{null}
\DeclareMathOperator{\spn}{span}
\DeclareMathOperator{\Mat}{Mat}
\let\hom\relax
\DeclareMathOperator{\hom}{Hom}
\DeclareMathOperator{\epi}{epi}
\DeclareMathOperator{\Sym}{Sym}
\DeclareMathOperator{\GL}{GL}
\DeclareMathOperator{\SL}{SL}
\DeclareMathOperator{\sgn}{sgn}
\DeclareMathOperator{\Aff}{Aff}
\DeclareMathOperator{\lcm}{lcm}
\DeclareMathOperator{\Isom}{\textbf{Isom}}
\DeclareMathOperator{\gen}{gen}
\DeclareMathOperator{\im}{im}
\DeclareMathOperator{\Aut}{Aut}
\DeclareMathOperator{\Inn}{Inn}
\DeclareMathOperator{\Sub}{\textbf{Sub}}
\DeclareMathOperator{\ind}{ind}
\DeclareMathOperator{\exterior}{Ext}
\DeclareMathOperator{\interior}{Int}
\DeclareMathOperator{\identity}{Id}
\DeclareMathOperator{\orthogonal}{O}
\DeclareMathOperator{\supp}{supp}
\DeclareMathOperator{\real}{Re}
\DeclareMathOperator{\imagine}{Im}
\DeclareMathOperator{\Ind}{Ind}
\DeclareMathOperator{\SRW}{SRW}
\DeclareMathOperator{\Markov}{Markov}
\DeclareMathOperator{\tr}{tr}
\DeclareMathOperator{\Poisson}{Poisson}
\DeclareMathOperator{\Geometric}{Geometric}
\DeclareMathOperator{\Exponential}{Exp}
\DeclareMathOperator{\cl}{cl}
\DeclareMathOperator{\Diff}{Diff}
\DeclareMathOperator{\ad}{ad}
\DeclareMathOperator{\Ad}{Ad}
\DeclareMathOperator{\unitary}{U}
\DeclareMathOperator{\End}{End}
\DeclareMathOperator{\inprod}{\lrcorner}
\DeclareMathOperator{\rot}{rot}
\DeclareMathOperator{\grad}{grad}
\DeclareMathOperator{\first}{I}
\DeclareMathOperator{\second}{II}
\DeclareMathOperator{\third}{III}
\DeclareMathOperator{\projection}{pr}
\DeclareMathOperator{\coker}{coker}
\DeclareMathOperator{\vol}{vol}
\DeclareMathOperator{\SO}{SO}


\def\*#1{\mathbf{#1}}
\newcommand{\veps}{\varepsilon}
\newcommand{\vphi}{\varphi}
\newcommand{\normsub}{\unlhd}
\newcommand{\snormsub}{\lhd}
\title{Maths 721 Notes}
\date{2020}

\begin{document}

\maketitle
\tableofcontents
\vspace{5mm}


% -------------------------------------------------------------------
% LECTURE 1
% -------------------------------------------------------------------

\pagebreak

\section{Lecture 1}


In the first half of this course we will cover three main topics:
\begin{itemize}
    \item representations;
    \item modules;
    \item characters.
\end{itemize}
We will further see that representations and modules are essentially the same, and that modules and characters are essentially the same; and hence all three are essentially the same.

From now on $G$ is a group.



\subsection{Representations}


\begin{definition}
A \textbf{representation} of a group $G$ over a field $F$ is a group homomorphism from $G$ to $\GL(n,F)$, where $n$ is the \textbf{degree} of the representation. 
\end{definition}

Explicitly, a representation is a function $\rho : G \to \GL(n,F)$ such that for all $g,h \in G$;
\begin{enumerate}[label=(\roman*)]
    \item $(gh)\rho = (g\rho)(h\rho)$,
    \item $1_G\rho = I_n$,
    \item $g^{-1}\rho = (g\rho)^{-1}$.
\end{enumerate}
Note the use of the (incredibly shit) postfix function notation.

\begin{example} \thlabel{d4}
Take $D_4$, the Dihedral group of order 8. It has the following group presentations
\begin{align*}
    D_4 &= \langle a,b \mid a^4 = 1, b^2 = 1, a^b = a^{-1} \rangle\\
    &\cong \langle (1 \; 2 \; 3 \; 4), (1 \; 4)(2 \; 3) \rangle,
\end{align*}
where $a^b = bab^{-1}$ is conjugation of $a$ by $b$. By defining the matrix subgroup
\[
    H = \left\langle A = \begin{bmatrix*}[r]
        0 & 1\\
        -1 & 0
    \end{bmatrix*}, B = \begin{bmatrix*}[r]
        1 & 0\\
        0 & -1
    \end{bmatrix*}\right\rangle
\]
and defining $\rho : D_4 \to \GL(n,F)$ where $F = \R, \C$, by $a \mapsto A$ and $b \mapsto B$, and
$a^ib^j \mapsto A^iB^j$ for $0 \le i \le 3$, and $0 \le j \le 1$. Hence we have $\rho$ is a representation of $D_4$ over $F$.
\end{example}

\begin{example}
Take $\Q_8$ the Quaternion group of order 8, which has the following group presentations
\begin{align*}
    \Q_8 &= \langle a,b \mid a^4 = 1, a^2 = b^2, a^b = a^{-1}\rangle\\
    &\cong \langle \bar a = (1 \; 6 \; 2 \; 5)(3 \; 8 \; 4 \; 7), \bar b = (1 \; 4 \; 2 \; 3)(5 \; 7 \; 6 \; 8)\rangle
\end{align*}
Define
\[
    H = \left\langle A = \begin{bmatrix*}[r]
        i & 0\\
        0 & -i
    \end{bmatrix*}, B = \begin{bmatrix*}[r]
        0 & 1\\
        -1 & 0
    \end{bmatrix*}\right\rangle \subset \GL(2,\C).
\]
Then $\rho : \Q_8 \to \GL(2,\C)$ defined by $a^kb^\ell \mapsto A^kB^\ell$ is a group representation of $\Q_8$ over $\C$ of degree 2.
\end{example}


\begin{definition}
Let $G$ be a group and define
\begin{align*}
    \rho : G &\to \GL(n,F)\\
    g\rho &= I_n
\end{align*}
for all $g \in G$. Then $\rho$ is a representation, called the \textbf{trivial representation} of arbitrary of degree.
\end{definition}

It follows from the trivial representation that any group $G$ has a representation of an arbitrary degree.

Let $\rho : G \to \GL(n,F)$ be a group homomorphism, and take $T \in \GL(n,F)$. Then
\[
    (T^{-1}AT)(T^{-1}BT) = T^{-1}(AB)T.
\]
Thus, given $\rho$ define $\sigma$ such that
\[
    g\sigma = T^{-1}(g\rho)T
\]
for all $g \in G$. As for all $g,h \in G$, one has
\begin{align*}
    (gh)\sigma &= T^{-1}\big((gh)\rho\big)T\\
    &= T^{-1}(g\rho)(h\rho)T\\
    &= T^{-1}(g\rho)TT^{-1}(h\rho)T\\
    &= (g\sigma)(h\sigma),
\end{align*}
and so $\sigma$ is a group homomorphism; and hence a representation.

\begin{definition}
Define
\[
    \rho : G \to \GL(m,F), \qquad \sigma : G \to \GL(n,F)
\]
to both be representation of $G$ over $F$. We say that \textbf{$\rho$ is equivalent to $\sigma$} if $n=m$ and there exists $T \in \GL(n,F)$ such that $g\sigma = T^{-1}(g\rho)T$.
\end{definition}

\begin{proposition}
Equivalence of representations is an equivalence relation.
\end{proposition}

\begin{proof}
Reflexivity is clear by taking $T = I_n$. For symmetry, take $T$ to be its inverse. For transitivity, if
\[
    g\sigma = T^{-1}(g\rho)T, \qquad g\rho = S^{-1}(g\eta)S,
\]
then
\[
    g\sigma = (ST)^{-1}(g\eta)(ST).
\]
\end{proof}

\begin{definition}
Define the \textbf{kernel} of the representation $\rho : G \to \GL(n,F)$ as $\ker \rho = \{g \in G \mid g\rho = I_n\}$.
\end{definition}

\begin{proposition}
The kernel of a representation of $G$ is a normal subgroup of $G$; i.e. $\ker \rho \lhd G$.
\end{proposition}

\begin{proof}
Suppose $g \in \ker \rho$ and $h \in G$ is arbitrary. Then
\[
    (hgh^{-1})\rho = (h\rho)(g\rho)(h^{-1}\rho) = (h\rho)I_n(h\rho)^{-1} = (h\rho)(h\rho)^{-1} = I_n,
\]
and so $hgh^{-1} \in \ker \rho$. As $\ker \rho$ is closed under conjugation, it is a normal subgroup of $G$.
\end{proof}

\begin{definition}
We say $\rho$ is a \textbf{faithful} representation of $G$ if $\ker \rho = \{1_G\}$. 
\end{definition}

\begin{example}
For the trivial representation $\rho : G \to \GL(n,F)$ with $g \mapsto I_n$ for all $g \in G$, we have $\ker \rho = G$. Hence the representation is not faithful.
\end{example}

\begin{lemma}
Suppose $G$ is a finite group, and $\rho$ is a representation of $G$ over $F$. Then $\rho$ is faithful if, and only if, $\im \rho \cong G$.
\end{lemma}

\begin{proof}
Immediate from the first isomorphism theorem.
\end{proof}



\subsection{$FG$-Modules}



Suppose $G$ is a group, and $F = \R,\C$. Given $\rho : G \to \GL(n,F)$, with $V = F^n$, let $v = (\lambda_1,\dots,\lambda_n) \in V$ for $\lambda_i \in F$ be a row vector. Moreover, note that $g\rho$ is an $n \times n$ matrix for all $g \in G$. Thus, we have $v \cdot (g\rho) \in V$, and satisfies the following properties:
\begin{enumerate}[label=(\roman*)]
    \item $v \cdot \big( (gh)\rho \big) = v \cdot (g\rho)(h\rho)$;
    \item $v \cdot (1_G\rho) = v$;
    \item $(\lambda v) \cdot (g\rho) = \lambda \big( v \cdot (g\rho) \big)$;
    \item $(u + v) \cdot (g\rho) = u \cdot (g\rho) + v \cdot (g\rho)$.
\end{enumerate}

We often will omitted the $\cdot$ in the operation, and write $v(a\rho)$ for $v \cdot (a\rho)$.

\begin{example}
Recall $D_4$ and its given presentation from a previous example. We have
\[
    a\rho = \begin{bmatrix*}[r]
        0 & 1\\
        -1 & 0
    \end{bmatrix*}, \qquad b\rho = \begin{bmatrix*}[r]
        1 & 0\\
        0 & -1
    \end{bmatrix*}.
\]
If $v = (\lambda_1,\lambda_2)$, then we have
\[
    v(a\rho) = (-\lambda_2,\lambda_1), \qquad v(b\rho) = (\lambda_1,-\lambda_2).
\]
\end{example}


\begin{definition}
Let $V$ a vector space over the field $F = \R,\C$, and let $G$ be a group. We say $V$ is a \textbf{$FG$-module} if a multiplication $v \cdot g$ for $v \in V$, and $g \in G$ is defined such that:
\begin{enumerate}[label=(\roman*)]
    \item $v \cdot g \in V$;
    \item $v \cdot (gh) = (v \cdot g) \cdot h$;
    \item $v \cdot 1_G = v$;
    \item $(\lambda v) \cdot g = \lambda(v \cdot g)$;
    \item $(u + v) \cdot g = u \cdot g + v \cdot g$.
\end{enumerate}
\end{definition}

This generalises the previous discussion from a matrix group to an arbitrary group.

Note that properties (i), (iv), and (v) imply that the map $v \mapsto v \cdot g$ is an endomorphism of $V$ (a linear map from $V$ to itself).

\begin{definition}
Suppose $V$ is an $FG$-module and $B$ is a basis for $V$. For $g \in G$, let $[g]_B$ denote the matrix of the endomorphism $v \mapsto v \cdot g$ of $V$ relative to the basis $B$.
\end{definition}




















% -----------------------------------------------------------
% LECTURE 2
% -----------------------------------------------------------



\section{Lecture 2}


\begin{theorem}
Let $\rho: G \to GL(n,F)$ be a representation of $G$ over $F$.
	\begin{enumerate}[label=\emph{(\Roman*)}]
	\item If $V=F^n$ is an $FG$ module and $G$ acts on $V$ by $v \cdot g = v (g\rho)$ there exists a basis $B$ of $V$ such that $g\rho = [g]_B$.
	\item The map $g \mapsto [g]_B$ is a representation for $G$ over $F$.
\end{enumerate}
\end{theorem}

\begin{proof}
%Continue
Choose the standard basis $B = [e_1, \ldots, e_n]$.

Since $V$ is an $FG$-module we have $v(gh) = (vg)h$ for all $g, h \in G$ and $v \in V$. Thus $[gh]_B = [g]_B [h]_B$ so the map is a homomorphism. We now check that $[g]_B$ is invertable for all $g \in G$. We know $v \cdot 1_G = (v g) g^{-1}$ so $I_n = [g]_B [g^{-1}]_B$ and thus $[g]_B$ has an inverse.
\end{proof}

\begin{example}
	Recall the representation of $G=D_4$ from a previous example. Define an $FG$-module $V=F^2$ with the action defined by taking $vg$ to $v(g\rho)$.
\begin{align*}
	v_1 &= (1,0), \quad v_1 a = v_2, \quad v_1 b = v_1,\\
	v_2 &= (0,1), \quad v_1 a = -v_1, \quad v_1 b = -v_2.
\end{align*}	
In this basis we recover our representation
\[
	a \mapsto [a]_B = \begin{bmatrix*}[r] 0 & 1\\ -1 & 0 \end{bmatrix*}, \quad
	b \mapsto [b]_B = \begin{bmatrix} 1 & 0\\ 0 & -1 \end{bmatrix}.
\]
\end{example}

%Similar exercise but for Q8.

We now provide an equivalent basis-dependent definition for an $FG$-module.
\begin{lemma}
	Let $V$ a vector space over the field $F = \R,\C$ with basis $B = [v_1, \ldots, v_n]$, and let $G$ be a group. If a multiplication $v \cdot g$ for $v \in B$, and $g \in G$ is defined such that:
\begin{enumerate}[label=(\roman*)]
    \item $v \cdot g \in V$;
    \item $v \cdot (gh) = (v \cdot g) \cdot h$;
    \item $v \cdot 1_G = v$;
    \item $\left( \sum_{i=1}^n \lambda_i v_i \right) \cdot g = \sum_{i=1}^n \lambda_i (v_i \cdot g)$ for all $\lambda_i \in F$;
\end{enumerate}
then $V$ is an $FG$-module.
\end{lemma}

%proof?

\begin{definition}
The trivial module of a group over $F$ is a one dimensional vector space $V$ over $F$ such that $v g = v$ for all $v \in V$ and $g \in G$.
\end{definition}

\begin{definition}
An $FG$-module is faithful if $1_G$ is the only $g \in G$ such that $v g = v$ for all $v \in V$.
\end{definition}

%Representations are equivelent for all choices of basis

\begin{theorem}
Let $V$ be an $FG$-module with basis $B$ and $\rho$ a representation of group $G$ over $F$ defined by taking $g \mapsto [g]_B$.
\begin{enumerate}[label=(\roman*)]
	\item If $B^\prime$ is another basis of $V$ then the map $g \mapsto [g]_{B^\prime}$ is a representation of $G$ equivalent to $\rho$.
	\item If representation $\sigma$ is equivalent to $\rho$ then there exists basis $B^{\prime\prime}$ such that $\sigma(g) = [g]_{B^{\prime\prime}}$ for all $g \in G$.
\end{enumerate}
\end{theorem}

\begin{proof}
Taking $T$ to be the change of basis matrices, the two representations are equivalent.
\end{proof}

\begin{example}
	Let $G = C_3 = \langle a \mid a^3 = 1 \rangle$ and representation $\rho: G \to GL(n,F)$ defined by
\[
	a \mapsto \begin{bmatrix} 0 & 1 \\ -1 & -1 \end{bmatrix}.
\]
	We attempt to construct an $FG$-module with group action described by $\rho$. Take $V = F^2$ with basis $B = [v_1, v_2]$. Define the action of $G$ on $V$ by
\[
	v_1 a = v_2, \quad v_2 a = -v_1 - v_2.
\]
	Let us now choose alternate basis $B^\prime = [u_1 = v_1, u_2 = v_1 + v_2]$. The action of $G$ on this basis is described by
\[
	u_1 a = -u_1 + u_2, \quad u_2 a = -u_1.
\]
This gives us a representation
\[
	a \mapsto [a]_{B^\prime} = \begin{bmatrix} -1 & 1\\ -1 & 0 \end{bmatrix}.
\]
To verify this construction we write our change of basis matrix as
\[
	T = \begin{bmatrix} 1 & 0\\ 1 & 1 \end{bmatrix}
\]
and verify that
\[
	\begin{bmatrix} 1 & 0\\ 1 & 1 \end{bmatrix}
	\begin{bmatrix} 0 & 1\\ -1 & -1 \end{bmatrix}
	\begin{bmatrix} 1 & 0\\ 1 & 1 \end{bmatrix}^{-1} = 
	\begin{bmatrix} -1 & 1\\ -1 & 0 \end{bmatrix}.
\]
\end{example}

\begin{definition}
	The permutation module of a group $G \leq S_n$ is an $n$-dimensional vector space $V$ with basis $B = [v_1, \ldots, v_n]$ and action by $G$ defined by
\[
	v_i g = v_{ig}
\]
for all $g \in G$ where $ig$ is the image of $i$ under $g \in S_n$.
\end{definition}

%Show that the permutation module is an FG-module.

It follows from Caley's theorem that every group has a faithful $FG$-module.

\begin{example}
Take $G = S_4$ and pick $g = (1\;2)$ and $h = (1\;2\;3\;4)$. We have representations
\[
    [g]_B = \begin{bmatrix} 0&1&0&0\\ 1&0&0&0\\ 0&0&1&0\\ 0&0&0&1 \end{bmatrix}, \quad
    [h]_B = \begin{bmatrix} 0&1&0&0\\ 0&0&1&0\\ 0&0&0&1\\ 1&0&0&0 \end{bmatrix}.
\]
\end{example}


\subsection{Module Reducibility}



\begin{definition}
Let $V$ be an $FG$-module. We call $W$ a submodule of $V$ if $W$ is a vector subspace of $V$ and $W$ is closed under the action of $G$. We then write $W<V$.
\end{definition}

\begin{example}
Let $G=C_3=\langle (1\;2\;3)\rangle$ and $V$ the permutation module of $G$ with basis $B=[v_1,v_2,v_3]$. The subspace $W=\langle v_1 + v_2 + v_3 \rangle$ is a submodule but the subspace $U=\langle v_1 + v_2 \rangle$ is not.

For example, consider the action of $g = (1\;2\;3)$ on $v_1 + v_2 \in U$. 
\[
	(v_1 + v_2)g = v_{1g} + v_{2g} = v_2 + v_3 \not\in U
\]
whereas $G$ acts on $W$ trivially.
\end{example}
















% ---------------------------------------------------------------------------------
% LECTURE 3
% ---------------------------------------------------------------------------------



\section{Lecture 3}

\subsection{Module and Representation Reducibility}

For any module, it is clear that we have two trivial submodules: $0<V$ and $V<V$. Where $0 = \{0\}\subset V$.

\begin{definition}
	Let $V$ be an FG-module. We say that $V$ is irreducible if the only submodules of $V$ are $V$ and $0$. Otherwise $V$ is reducible
\end{definition}

In 2.11 we showed that the permutation module of $C_3$ is reducible.

\begin{definition}
	Let $\rho:G\rightarrow GL(n,F)$ be a representation. We say that $\rho$ is irreducible if the corresponding $FG$-module (as constructed in 2.1) is irreducible. Otherwise $\rho$ is reducible.
\end{definition}

If an $FG$-module, $V$  is reducible, that is, $0<W<V$, $0\neq W\neq V$. Let $B_W$ be a basis for $W$. If we extend $B_W$ to $B$ a basis of $V$, then we get the following representation of $G$:

\begin{equation}\label{eq:3.1}
	g\mapsto [g]_B = \begin{bmatrix}
		X_g & 0\\Y_g & Z_g
	\end{bmatrix}
\end{equation}
Where the matrices $X_g,Y_g$ and $Z_g$ are some block matrices and $0$ is a block of zeros and $X_g$ has the dimensions $m\times m$ and, in this case, $\operatorname{dim}(W)=m$.
	
\begin{proposition}
	A representation $\rho:G\rightarrow GL(n,F)$ is reducible if and only if with respect to some basis, $B$, of $F^n$, $[g]_B$ has the form \ref{eq:3.1} for some $0<m<\operatorname{dim}(V)$ for all $g\in G$. Then the maps $g\mapsto X_g$ and $g\mapsto Z_g$ are both representations of $G$.
\end{proposition}

\begin{proof}
	Suppose we have a presentation, $\rho:G\rightarrow GL(n,F)$ and a basis $B$ of $V = F^n$ such that $[g]_B$ has the form \ref{eq:3.1} for every $g\in G$. Then consider the subspace $0\subset W\subset V$ spanned by the first $m$ elements of $B$. It is clear that $v[g]_B\in W$ for all $v\in W$. Therefore the module induced by $\rho$ is reducible, so $\rho$ is reducible. Now, if we have a reducible representation, then the argument above this proposition shows that with respect to any basis extending $B_W$, the matrices $[g]_B$ have the required form.\\
	Now, using elementary block matrix multiplication, we get the following for $g,h\in G$: 
	\[\rho(g)\rho(h) = [g]_B[h]_B = \begin{bmatrix}
	X_gX_h & 0\\
	Y_gX_h+Z_gY_h & Z_gZ_h
	\end{bmatrix} = [gh]_B=\rho(gh)\]
	Therefore $X_{gh} = X_gX_h$ and $Z_{gh} = Z_gZ_h$, so the maps $g\mapsto X_g$ and $g\mapsto Z_g$ are both representations of $G$.
\end{proof}

\begin{problem}
	Prove that the example representation of $D_8$ of degree 2 over $\R$ or $\C$ is irreducible.
\end{problem}

\subsection{Group Algebras}

Recall that an algebra over a field $F$ is a  vector space over $F$ equipped with a bilinear product $A\times A\rightarrow A$ that is compatible with scalar multiplication.

\begin{definition}
	The group algebra over a finite group $G$ over a field $F$ is an algebra\footnote{See Lemma\ref{le:3.6} } of dimension $n = |G|$ over $F=\R$ or $\C$ called $FG$, with basis $B=G = \{g_1,\dots g_n\}$. Where the algebra structure is given by the following for two arbitrary elements of $FG$, $u = \sum_{g\in G}\lambda_gg$, $v = \sum_{g\in G}\mu_g$, $\lambda_g,\mu_g\in F$ and $\nu\in F$:
	\begin{enumerate}[label=(\roman*)]
		\item $u+v = \sum_{i=1}^n(\lambda_i+\mu_i)g_i$
		\item $\nu \cdot u = \sum_{i=1}^{n}(\nu\lambda_i)g_i$
		\item $u\cdot v = \sum_{(h,g)\in G\times G}\lambda_g\mu_h(gh)$
	\end{enumerate}
\end{definition}

This is clearly a vector space.

\begin{example}
	Consider $G = C_3 = \{e,a,a^2\} = \langle a|a^3=e\rangle$ and $F=\R$ or $\C$. Then if we let $u = e-a+2a^2$, $v=\frac{1}{2}e+5a$, then:
	\[u+v = \frac{3}{2}e+4a+2a^2,\quad \frac{1}{3}u = \frac{1}{3}e-\frac{1}{3}a+\frac{2}{3}a^2,\quad uv = \frac{21}{2}e+\frac{9}{2}a-4a^2\]
\end{example}
\begin{lemma}\label{le:3.6}
	Given a group algebra $FG$, $r,s,t\in FG$, $\lambda\in F$:
	\begin{enumerate}[label=\emph{(\Roman*)}]
		\item $rs\in FG$
		\item $(rs)t = r(st)$
		\item $1_Gr=r1_G=r$
		\item $(\lambda r)s=\lambda(rs)$
		\item $(r+s)t=rt+st$
		\item $r(s+t)=rs+rt$
		\item $r0=0r=0$
	\end{enumerate}
That is, $FG$ is an associative algebra with unit
\end{lemma}
\begin{proof}
	1,3 and 7 are clear from the definition of $FG$, 4,5 and 6 follow from the distributive and associative laws of $F$ and 2 follows from associativity in $G$.
\end{proof}
\subsection{The Regular FG-module, FG}
\begin{problem}
	$V=FG$ is an $FG$-module with the group action defined by $v\cdot g=vg$ for $v\in FG$, $g\in G\subset FG$.
\end{problem}
\begin{definition}
	For a finite group $G$ and $F=\R$ or $\C$, the regular $FG$-module is $FG$. The associated module, $g\mapsto [g]_B$ is called the regular representation.
\end{definition}
\begin{lemma}
	$FG$ is a faithful module for $G$ over $F$
\end{lemma}
\begin{proof}
	If $vg=v$ for all $v\in FG$, then specifically, $hg=h$ for all $h\in G$, so $g=1_G$.
\end{proof}
\begin{example}
	For $C=C_3$, over the basis $B=G$, we get:
	\[[e]_B=I_3,\quad [a]_B = \begin{bmatrix}
	0&1&0\\0&0&1\\1&0&0
	\end{bmatrix},\quad[a^2]_B=\begin{bmatrix}
	0&0&1\\1&0&0\\0&1&0
	\end{bmatrix}\]
\end{example}
Now, if we have an $FG$-module, $V$, then $FG$ acts on $V$ in the following way:
\[v\cdot r = v\cdot \left(\sum_{g\in G}\mu_g g\right) = \sum_{g\in G}\mu_g(v\cdot g)\]
\begin{lemma}\label{Lem:FG}
	For $u,v\in V$, $\lambda\in F$, $r,s\in FG$:
	\begin{enumerate}[label=\emph{(\Roman*)}]
		\item $vr\in FG$
		\item $(vr)s = v(rs)$
		\item $v1=v$
		\item $(\lambda v)r=\lambda(vr)=v(\lambda r)$
		\item $v(r+s)=vr+vs$
		\item $(u+v)r=ur+vr$
		\item $r0=v0=0$
	\end{enumerate}
\end{lemma}
\begin{proof}
	I,III and the first part of VII follow from $V$ being an $FG$-module, the second equality of VII follows from scalar multiplication by 0 in $V$. The following calculation:
	\begin{align*}
	(\lambda v)r &= \sum_{g\in G}\mu_g((\lambda v)g)\\
	&=\sum_{g\in G}\mu_g(\lambda (vg))\\
	&=\sum_{g\in G}(\lambda \mu_g)(vg)=v(\lambda r)\\
	&=\lambda\sum_{g\in G}\mu_g (vg)\\
	&=\lambda(vg)\\
	\end{align*}proves IV. VI follows from the linearity of the action of $G$ on $V$. V follows from distributivity of scalar multiplication in $V$. Finally, to prove II:
	\begin{align*}
	v(rs)&=\sum_{(g,h)\in G\times G}(\mu_g\lambda_h(v(gh)))\\
	&= \sum_{h\in G}\lambda_h\sum_{g\in G}\mu_g(gv)h\\
	&= \sum_{h\in G}\lambda_h\left(\sum_{g\in G}\mu_g(gv)\right)h\\
	&=\sum_{h\in G}\lambda_h(vr)h=(vr)s
	\end{align*}
\end{proof}

















% -------------------------------------------------------------------
% LECTURE 4
% -------------------------------------------------------------------
\section{Lecture 4}

\subsection{Homomorphisms} 

\begin{definition}
	Let $V$ and $W$ be $FG$-modules. A \textit{homomorphism} of $FG$-modules is a map $\sigma: V \rightarrow W$ which is a linear transformation and also satisfies $(vg)\sigma = (v\sigma)g$ for all $g\in G, v\in V$. The \textit{kernel} and \textit{image} are defined in the obvious way
\end{definition}
Equivalently, it is a homomorphism of modules over the ring $FG$. Indeed:
\begin{problem}\thlabel{module-hom}
	Suppose $r\in FG$ is an element of the group algebra. Prove that $(vr)\sigma = (v\sigma)r$.
\end{problem}



\begin{lemma}
	Let $\sigma: V \rightarrow W$ be a homomorphism of $FG$-algebras. Then the kernel and image of $\sigma$ are submodules
\end{lemma}
\begin{proof}
	This is a matter of simple checking, which will be left to the reader.
\end{proof}

\begin{example}
	Take $\sigma: V \rightarrow V$ to be $v \mapsto \lambda v$ for some $\lambda \in F^*$. Then $\ker \sigma = 0, \im \sigma = V$. 
\end{example}
\begin{example}
	Let $G = S_n$ and $V = \langle v_1,..., v_n \rangle$ be the permutation module for $G$ over $F$, and let $W = \langle w \rangle$ be the trivial module. Now define $\sigma: V \rightarrow W$ by
\[
	\sum \lambda_i v_i \mapsto \sum \lambda_i w
\]
Then $\ker \sigma = \{\sum \lambda_i v_i \mid \sum \lambda_i = 0\}$ and $\im \sigma = W$. 
\end{example}

\begin{definition}
	A homomorphism of $FG$-modules is an \textit{isomorphism} if it is bijective
\end{definition}
\begin{remark}
	In class we originally said "if the homomorphism has trivial kernel". However, this is definitely not correct because inclusions are always homomorphisms, but obviously not isomorphisms. 
\end{remark}

\begin{lemma}
	The inverse of an isomorphism is an isomorphism
\end{lemma}
\begin{proof}
	Once again, this is just an exercise in checking. The details will be left for the reader.
\end{proof}
Some rather obvious invariants of $FG$-modules (under isomorphism) are dimension and irreducibility.
\begin{lemma}
	$V$ and $W$ are isomorphic if and only if there exists bases $\mathcal{B}_1$ of $V$ and $\mathcal{B}_2$ of $W$ such that
\[
	[g]_{\mathcal{B}_1} = [g]_{\mathcal{B}_2}
\]
for all $g$.
\end{lemma}
\begin{proof}
	Suppose firstly that $V$ and $W$ are isomorphic, and let $\sigma: V \rightarrow W$ be one such isomorphism. Let $\mathcal{B}_1 = \{v_1,...,v_n\}$ be a basis for $V$. In particular, it is linearly independent, and it is easy to see that $\mathcal{B}_2 = \{v_1\sigma,...,v_n\sigma\}$ is also linearly independent. Since $V$ and $W$ are isomorphic, they have the same dimension, and thus $\mathcal{B}_2$ is a basis for $W$. Since $(vg)\sigma = (v\sigma)g$ for all $g$ and $v$, the action of $g$ on the basis vectors of both bases are the same, and thus we conclude $[g]_{\mathcal{B}_1} = [g]_{\mathcal{B}_2}$.
	\\\\
	Conversely, suppose that the latter hypothesis is satisfied. Let $\{v_1,...,v_n\}$ be a basis for $V$ and $\{w_1,...,w_n\}$ be a basis for $W$. We define a bijective linear map $\sigma: V \rightarrow W$ such that $v_i \sigma = w_i$ for each $i$. Now observe that for each $i$, we have  $v_ig = \lambda_1 v_1 + ... + \lambda_n v_n$ and $w_ig = \lambda_1 w_1 +...+ \lambda_n w_n$, where $(\lambda_1,...,\lambda_n)$ is the $i$-th row of $[g]$. This means that
\[
	(v_ig)\sigma = (\lambda_1 v_1 + ... + \lambda_n v_n)\sigma = \lambda_1 v_1\sigma + ... + \lambda_n v_n\sigma = \lambda_1 w_1 +...+ \lambda_n w_n = w_ig = (v_i\sigma)g
\]
and thus $\sigma$ is a homomorphism of $FG$-modules. Since it is bijective, it is an isomorphism.
\end{proof}
\begin{theorem}
	Let $V$ be an $FG$-module with basis $\mcl{B}_1$ and $W$ an $FG$-module with basis $\mcl{B}_2$. Then $W \cong V$ if and only if $g\mapsto [g]_{\mcl{B}_1}$ and $g\mapsto [g]_{\mcl{B}_2}$ are equivalent.
\end{theorem}
\begin{proof}
	This follows from the previous Lemma and the fact that two matrices are conjugate ($A$ and $B$ are conjugate if $A = P^{-1}B P$ for some $P$) if and only if the linear transformations they define differ by a change of basis (that is they define the same transformation but with respect to different bases)
\end{proof}

\begin{example}
	Let $G = C_3 = \{e, a, a^2\}$. Let $V$ be the regular representation, that is the natural representation induced by the module $FG = \langle e, a, a^2 \rangle$. Write $B:= \{e, a, a^2\}$ as a basis for $FG$. Then
\[
	[a]_B = \left(\begin{matrix}
	0 & 1 & 0 \\ 0 & 0 & 1 \\ 1 & 0 & 0
	\end{matrix}\right)
\]
Now let $W$ be the permutation module where $a = (1, 2, 3)$ and $C_3$ is considered a subgroup of $S_3$. Write $B' = \{v_1, v_2, v_3\}$ for the basis of $W$. Then
\[
	[a]_{B'} = \left(\begin{matrix}
	0 & 1 & 0 \\ 0 & 0 & 1 \\ 1 & 0 & 0
	\end{matrix}\right)
\]
	Note that these two modules are isomorphic. 
\end{example}
\begin{example}
	Let $G = D_4 = \langle a, b \mid a^4 = b^2 = 1, a^b = a^{-1}\rangle$. Now we can act on either $F^4$ or $F^8$. On $F^4$, we have the representation described in \thref{d4}. On $W$, we have the regular representation. Clearly are not isomorphic.
\end{example}

\subsection{Sums}

We now consider how modules behave with respect to direct sums. Let $V$ be an $FG$-module and suppose $V = U \oplus W$, where $U$ and $W$ are submodules. Let $\mcl{B}_1 =\{u_1,...,u_n\}$ be a basis for $U$ and $\mcl{B}_2 =\{w_1,...,w_m\}$ one for $W$, so that $\mcl{B} = \mcl{B}_1 \cup \mcl{B_2}$ is a basis for $V$. Then 
\[
    [g]_{\mcl{B}} = \begin{pmatrix*}
        [g]_{\mcl B_1} & 0 \\0 & [g]_{\mcl{B}_2}
    \end{pmatrix*}
\]
%(I have no idea why, but the moment I try use square brackets around that $g$ it fucks up - Oliver). 

\begin{lemma}
	Let $V$ be an $FG$-module such that we have the decomposition
\[
	V = \bigoplus_{i = 1}^n U_i
\]
Define the projection map $\pi_i: u_1 + u_2 +...+ u_n \mapsto u_i$. Then 
	\begin{enumerate}[label=\emph{(\Roman*)}]
		\item $\pi_i$ is a homomorphism
		\item $\pi_i \circ \pi_i = \pi_i$
	\end{enumerate}
\end{lemma}
\begin{proof}
	Trivial
\end{proof}

\begin{lemma}
	Suppose we have a finite decomposition
\[
	V = \sum U_i
\]
where the $U_i$ are irreducible. Then $V$ is the direct sum of some subset of the $U_i$. 
\end{lemma}
\begin{proof}
	This follows from the fact that the intersection of two distinct irreducible modules is trivial (again, simple checking).
\end{proof}

We will now present an important result
\begin{theorem} [Maschke's Theorem]
	Let $G$ be a finite group, $F$ a field of characteristic 0, $V$ an $FG$-module and $U$ a submodule. Then there exists some $W$ such that $V = U \oplus W$.
\end{theorem}
\begin{proof}
	We first choose some $W_1$ such that $V = U \oplus W_1$ as vector spaces. Note that each $v\in V$ can be uniquely decomposed as $v = u + w$ , where $u\in U, w\in W_1$. Now define the canonical projection $\sigma: V \rightarrow U$ where $v \mapsto u$. Clearly $\ker \sigma = W_1$ and $\im \sigma = U$. However, we note that $\sigma$ is NOT necessarily a homomorphism of $FG$-modules. We modify it as follows: Define $\varphi: V \rightarrow V$ by
\[
	v\mapsto \frac{1}{{|G|}} \sum_{g\in G} vg\sigma g^{-1}
\]
We claim that $\varphi$ IS a homomorphism. Indeed, suppose $x\in G, v\in V$. Then 
	\begin{align*}
		(xv)\varphi &= \frac{1}{{|G|}} \sum_{g\in G} (vx)g\sigma g^{-1}\\
		&= \frac{1}{{|G|}} \sum_{h\in G} vh\sigma h^{-1}x \\
		&= \left(\frac{1}{{|G|}} \sum_{h\in G} vh\sigma h^{-1}\right)x = (v\varphi)x
	\end{align*}
	where the equality
\[
	\frac{1}{{|G|}} \sum_{g\in G} (vx)g\sigma g^{-1}= \frac{1}{{|G|}} \sum_{h\in G} vh\sigma h^{-1}x
\]
follows from the change of variables $h = xg$. Clearly $\varphi$ maps into $U$, and we now check it is a projection. Indeed, supposing $u\in U$ we have 
	\begin{align*}
		(u)\varphi &= \frac{1}{{|G|}} \sum_{g\in G} ug\sigma g^{-1} \\
		&= \frac{1}{{|G|}} \sum_{g\in G} u\sigma gg^{-1} \\
		&= \frac{1}{{|G|}} \sum_{g\in G} u\\
		&= u
	\end{align*}
	as desired. 
	\\\\
	Now clearly $U = \im \varphi$ and we define $W:= \ker \varphi$. Then for each $v\in V$, write $u:= v\varphi \in U$ and $w:= v - u\in W$ so that $v = u + w$. It only remains to check that this is unique. To see this, suppose $u' + w' = v = u + w$. Then
\[
	u' = \varphi(u') = \varphi(v) = \varphi(u) = u
\]
which implies the result.
\end{proof}










% -------------------------------------------------------------------
% LECTURE 5
% -------------------------------------------------------------------



\section{Lecture 5}



We begin with consequences of Maschke's theorem.

\begin{example}
Let $G = S_3$, and $V = \langle v_1,v_2,v_3 \rangle$ is the permutation module. Let $U$ be the submodule of $V$ defined as $U = \langle v_1 + v_2 + v_3 \rangle < V$. Suppose $W_0 = \langle v_1, v_2 \rangle$, so that $V = U \oplus W_0$ as subspaces. Define a projection $\phi : V \to U$ by $v_1 \mapsto 0$, $v_2 \mapsto 0$, and $v_3 \mapsto v_1 + v_2 + v_3$. Further, define $\theta : V \to V$, as in proof of Maschke's theorem, so that
\[
    v\theta = \frac{1}{|G|}\sum_{g \in G}vg\phi g^{-1} = \frac{1}{6}\sum_{g \in S_3}vg\phi g^{-1}.
\]
Consider the action of $\theta$ on the basis elements $v_i$, then a short computation shows that
\[
    v_i\theta = \frac{1}{3}(v_1 + v_2 + v_3), \qquad i=1,2,3.
\]
Moreover, we have
\[
    \ker \theta = \{v \in V : v\theta = 0\} = \bigg\{\sum_{i=1}^3 \lambda_iv_i : \sum_{i=1}^3 \lambda_i = 0\bigg\}.
\]
Hence $V = U \oplus \ker \theta$, is a direct summand of submodules. Moreover, if $\mathcal B = [v_1+v_2+v_3,v_1,v_2]$ is a basis for $V$, and $\mathcal B' = [v_1 + v_2 + v_3, v_1 - v_2, v_2 - v_3]$ is another, one has
\[
    [g]_{\mathcal B} = \begin{pmatrix*}
        * & 0 & 0\\
        * & * & *\\
        * & * & *
    \end{pmatrix*}, \qquad [g]_{\mathcal B'} = \begin{pmatrix*}
        * & 0 & 0\\
        0 & * & *\\
        0 & * & *
    \end{pmatrix*}.
\]
\end{example}

In fact, it follows from Maschke's theorem that if we choose a basis $\mathcal B$ for $V$ such that
\[
    [g]_{\mathcal B} = \begin{pmatrix*}
        * & 0\\ * & *
    \end{pmatrix*},
\]
then there exists a basis $\mathcal B'$ for $V$ such that
\[
    [g]_{\mathcal B'} = \begin{pmatrix*}
        * & 0\\0 & *
    \end{pmatrix*}.
\]

\begin{definition}
Let $V$ be a $FG$-module, $V$ is said to be \textbf{completely reducible} if $V = U_1 \oplus \cdots \oplus U_r$ with each $U_i$ an irreducible $FG$-module.
\end{definition}

\begin{theorem}
Suppose $G$ is a finite group, and $F = \R,\C$. Then every $FG$-module is completely reducible
\end{theorem}

\begin{proof}
Induction using Maschke's theorem.
\end{proof}

\begin{lemma}
Suppose $G$ is a finite group, $F = \R,\C$, and $V$ a $FG$-module. If $U$ is a $FG$-submodule, then there exists a surjective $FG$-homomorphism from $V$ onto $U$.
\end{lemma}

\begin{proof}
By Maschke's theorem, there exists a complementary submodule $W$ to $U$ such that $V = U \oplus W$. Thus, defining $\pi : V \to U$ by $u + w \mapsto u$ gives the result.
\end{proof}


\begin{example}
Take
\[
    G = \left\langle \begin{pmatrix*}
        1 & 0 \\ n & 1
    \end{pmatrix*} \mid n \in \Z \right\rangle
\]
and $V = \C^2$. Then $V$ is not completely reducible.
\end{example}


\begin{example}
Let $G = C_p = \langle a \mid a^p = 1 \rangle$ where $p$ is prime, and take the representation
\[
    a^j \mapsto \begin{pmatrix*}
        1 & j\\ 0 & 1
    \end{pmatrix*}, \qquad 0 \le j \le p-1
\]
over the finite field $F = \Z_p$. If $V = \langle v_1, v_2 \rangle$ is the $FG$-module of the representation and $U= \langle v_2 \rangle$, then there does not exist a submodule $W < V$ such that $V = U \oplus W$.
\end{example}

The previous two examples show that the assumptions of Maschke's theorem are required, and cannot be relaxed.


\subsection{Schur's Lemma}

\begin{theorem}[Schur's lemma]
Suppose $V$ and $W$ are irreducible $\C G$-modules.
\begin{enumerate}[label=\emph{(\Roman*)}]
    \item If $\theta : V \to W$ is a $\C G$-homomorphism, then $\theta$ is either a $\C G$-isomorphism or the zero homomorphism.
    
    \item If $\theta : V \to V$ is a $\C G$-isomorphism, then $\theta$ is scalar multiple of the identity endomorphism of $V$.
\end{enumerate}
\end{theorem}

\begin{proof}
(I): Suppose that $v\theta \neq 0$ for some $v \in V$, then the image of $\theta$ is not trivial, $\im \theta \neq \{0\}$. However, $\im\theta$ is a submodule of $W$, and so the irreducibility of $W$ forces $\im \theta = W$. Likewise, as the kernel of $\theta$ is a submodule of $V$, but not all of $V$, we have $\ker \theta = \{0\}$ as $V$ is irreducible. Therefore, $\theta$ is a bijective $\C G$-homomorphism, and so it is a $\C G$-isomorphism.

(II): Suppose $\theta$ is a $\C G$-isomorphism. Then as $\C$ is algebraically closed, $\theta$ has an eigenvalue $\lambda_v \in \C$ with $v \theta = \lambda_v v$ for some $v \in V$. Now, as $\ker(\theta - \lambda_v 1_V) \neq \{0\}$ is a submodule of $V$, $V$ being irreducible implies that $\ker(\theta - \lambda_v1_V) = V$. Therefore, $w(\theta - \lambda_v1_V) = 0$ for all $w \in V$, and so $\theta = \lambda 1_V$ as required. 
\end{proof}

Further, we actually have a converse to the second statement of Schur's lemma.

\begin{proposition}
Let $V$ be a nontrivial $\C G$-module, and suppose that every $\C G$-homomorphism from $V$ to $V$ is a scalar multiple of the identity endomorphism of $V$. Then $V$ is irreducible.
\end{proposition}

\begin{proof}
Suppose $V$ is reducible. Then there exists a nontrivial submodule $U$ of $V$ such that, by Maschke's theorem, $V = U \oplus W$ with $W$ also a submodule of $V$. Defining $\pi : V \to V$ by $u + w \mapsto u$ gives a $\C G$-homomorphism that is not a multiple of the identity endomorphism. A contradiction.
\end{proof}


We now interpret Schur's lemma as representation statement.

\begin{lemma}
Let $\rho : G \to \GL(n,\C)$ be a representation. Then $\rho$ is irreducible if, and only if, every $n \times n$ matrix $A$ which satisfies $(g\rho)A = A(g\rho)$ for all $g \in G$, has the form $A = \lambda I_n$.
\end{lemma}

\begin{proof}
Result follows from Schur's lemma and its partial converse.
\end{proof}

\begin{example}
Suppose $G = C_3 = \langle a \mid a^3 = 1\rangle$, and $\rho : G \to \GL(2,\C)$ is the representation defined by
\[
    a\rho = \begin{pmatrix*}[r]
        0 & 1\\ -1 & -1
    \end{pmatrix*}.
\]
Let
\[
    A = a\rho = \begin{pmatrix*}[r]
        0 & 1\\ -1 & -1
    \end{pmatrix*}.
\]
Then it is clear that $A(g\rho) = (g\rho)A$ for all $g \in G$. As $A$ is not a scalar multiple of the identity, the representation is reducible.
\end{example}

\begin{example}
Suppose $G = D_5 = \langle a,b \mid a^5 = 1, b^2=1, a^b=a^{-1}\rangle$. Set $\omega = e^{2\pi i/5}$. Let $\rho : G \to \GL(2,\C)$ be the representation defined by
\[
    a \mapsto \begin{pmatrix*}[r]
        \omega & 0\\ 0 & \omega^{-1}
    \end{pmatrix*}, \qquad b \mapsto \begin{pmatrix*}[r]
        0 & 1\\ 1 & 0
    \end{pmatrix*}.
\]
If
\[
    A = \begin{pmatrix*}[r]
        \alpha & \beta \\ \gamma & \delta
    \end{pmatrix*}
\]
commutes with $a \rho$ and $b \rho$; one calculates that
\[
    A = \begin{pmatrix*}[r]
        \alpha & 0 \\ 0 & \alpha
    \end{pmatrix*} = \alpha I_n,
\]
and so the representation is irreducible.
\end{example}


\begin{lemma}
Let $G$ be a finite abelian group. Then every irreducible $\C G$-module has dimension 1.
\end{lemma}

\begin{proof}
Choose $x \in G$, then $v(gx) = v(xg)$ for all $g \in G$, and so $v \mapsto vx$ is a $\C G$-homomorphism. It is actually an isomorphism with inverse $v \mapsto vx^{-1}$. Hence by Schur this isomorphism is a scalar multiple of the identity, say $\lambda_x1_V$. Thus $vx = \lambda_x1_V$ for all $v \in V$ and the group action by $G$ is just usual scalar multiplication. This means that every subspace is a submodule. However, $V$ is irreducible and so has no non-trivial submodules; which forces $\dim V = 1$.
\end{proof}









% -------------------------------------------------------------------
% LECTURE 6
% -------------------------------------------------------------------


\section{Lecture 6}


Continuing our discussion of the representations of abelian groups, we provide a stronger theorem in which we construct these 1-dimensional representations by mapping group elements to roots of unity. But first recall the Fundamental Theorem of Abelian Groups which states that any finite abelian group $G$ is isomorphic to the direct sum of cyclic groups
\[
	G \simeq C_{n_1} \oplus \cdots \oplus C_{n_r}
\]
where $n_i \mid n_{i+1}$ for all $1 \leq i \leq r-1$. Note that if $C_i = \langle g_i \rangle$ then we can write $G = \langle g_1,\ldots, g_r \rangle$ and $g_i$ has order $n_i$.

We define a homomorphism $\rho$ from $G$ to $\mathbb{C}$ by taking $g_i \mapsto \lambda_i$ where $\lambda_i$ is the $n_i$-th root of unity. This defines a representation and is specified by roots of unity $\lambda_1,\cdots, \lambda_r$. Thus for the representation $\rho$ defined by roots of unity $\lambda_1,\ldots, \lambda_r$ we write $\rho = \rho_{\lambda_1,\ldots,\lambda_r}$.
%Conversely?

\begin{theorem}
Suppose $G\simeq C_{n_1}\times\cdots\times C_{n_r}$ for cyclic groups $C_{n_i}$ of order $n_i$. The representation $\rho_{\lambda_1,\ldots,\lambda_r}$ of $G$ is irreducible of degree 1. There are $|G|$ many such representations and every irreducible representation of $G$ over $\mathbb{C}$ is equivalent to one of these.
\end{theorem}

%Prove theorem

\begin{example}
Take $G=\langle a \mid a^n = 1 \rangle$ and $\omega = e^{2\pi i/n}$. The irreducible representations of $G$ are $\rho_{\omega^j}$ for $0 \leq j \leq n-1$ defined by
\[
	a^k \rho_{\omega^j} = \omega^{jk}, \quad 0 \leq k \leq n-1.
\]
\end{example}

%Klein-4 group example

%Diagonalisation?
%\subsection{Diagonalisation}
%
%\begin{lemma}
%	Let $G = \langle g\rangle$ be a cyclic group of order $n$ and take $V$ a $\mathbb{C}G$-module. We can write $V = U_1 \oplus\cdots\oplus U_r$, $U_i$ irreducible of dimension 1. Take $\omega = 2^{2\pi i/n}$ define the action of $G$ on $V$ bu $u_i\cdot g = \omega^{m_i}u_i$ for each $i$. There exists a basis $B=(u_1,\ldots,u_r)$ such that $[[g]_B$ is diagonal.
%\end{lemma}



\subsection{Application of Schur's to $\mathbb{C}G$}

\begin{definition}
If $G$ is a finite group then we define the center of the group algebra $Z(\mathbb{C}G)$ by
\[
	Z(\mathbb{C}G) = \{ z \in\mathbb{C}G \mid zr = rz \quad\forall r\in\mathbb{C}G \}.
\]
\end{definition}

We now state some simple properties of the center:
\begin{enumerate}
	\item $Z(\mathbb{C}G)$ is a subspace of $\mathbb{C}G$.
	\item If $G$ is abelian then $Z(\mathbb{C}G) = \mathbb{C}Z(G)$.
	\item If $H$ is a normal subgroup of $G$ then $\sum_{h \in H} h \in Z(\mathbb{C}G)$.
\end{enumerate}
We provide a simple proof for the last statement.
\begin{proof}
Take $z=\sum_{h \in H} h$ and $g\in G$. We then have
\[
	z^g = \sum_{h \in H} h^g = z.
\]
because $H$ is normal and hence fixed by conjugation of elements in $G$. 
\end{proof}

\begin{example}
	Take $G = S_3 = \langle a=(1\;2\;3), b=(1\;2) \rangle$. Then $\sum_{g\in G} \in Z(\mathbb{C}G)$.
\end{example}
%What is this example even?

\begin{lemma}
Let $V$ be an irreducible $\mathbb{C}G$-module and let $z \in Z(\mathbb{C}G)$. There exists $\lambda \in \mathbb{C}$ such that $vz = \lambda v$ for all $v\in V$.
\end{lemma}
\begin{proof}
For all $r\in\mathbb{C}G$ and $v\in V$ we know $vrz = vzr$. Thus the mapping $v\mapsto vz$ is a $\mathbb{C}G$-homomorphism from $V$ to $V$. The result then follows by Schur.
\end{proof}

\begin{remark}
Note that $\mathbb{C}G$ is a faithful module.
\end{remark}

\begin{lemma}
If there exists a faithful, irreducible $\mathbb{C}G$-module then $Z(G)$ is cyclic.
\end{lemma}
\begin{proof}
Let $V$ be an irreducible, faithful $\mathbb{C}G$-module and take $z\in Z(G)\subset Z(\mathbb{C}G)$. There exists $\lambda_z\in\mathbb{C}$ such that $vz = \lambda_z v$ for all $v\in V$. But since $V$ is faithful the mapping $z\mapsto \lambda_z$ is injective from $Z(G)$ to $\mathbb{C}^\times$ and so $Z(G) \simeq \{\lambda_z \mid z\in Z(G)\}$ is a finite subgroup of $\mathbb{C}^\times$. All of which are cyclic.
\end{proof}

\begin{example}
$\mathbb{Z}_4$ has a faithful, irreducible representation by taking 1 to the 4-th root of unity. However $\mathbb{Z}_2\times\mathbb{Z}_2$ has no faithful irreducible representation as its center is not cyclic.
\end{example}

\begin{remark}
Note that the converse is not true in general. There exists groups with cyclic centers but no faithful, irreducible representations. See Frobenius Groups.
\end{remark}

%State the converse
\begin{lemma}
Suppose $G$ finite and every irreducible $\mathbb{C}G$-module has dimension 1. Then $G$ is abelian.
\end{lemma}
\begin{proof}
	Since $\CG$ is is a $\CG$-module and it is completely reducible, we can decompose it as follows \[\CG = \bigoplus_{i = 1}^{|G|} V_i\] where each $V_i$ is one-dimensional. For any $v_i\in V_i$ and $x,y\in G$, note that \[v_ixy = \lambda_xv_iy = \lambda_y\lambda_xv_i = \lambda_x\lambda_yv_i = v_iyx\] Since the $v_i$ form a basis for $\CG$ this means $vxy = vyx$ for all $v\in \CG$ and $x,y\in G$. Since $\CG$ is faithful, the result follows.
\end{proof}

\subsection{The Group Algebra and Irreducible Modules}


\begin{lemma}\label{Lem:FIT}
Let $V,W$ be $\mathbb{C}G$-modules and $\theta : V \to W$ is a $\mathbb{C}G$-homomorphism. There exists a submodule $U < V$ such that $V = \ker\theta \oplus U$ and $U \simeq \im\theta$
\end{lemma}
\begin{proof}
We apply Maschke's theorem to $\ker\theta < V$ and obtain submodule $U < V$ such that $V = \ker\theta \oplus U$. Now define the map $\overline\theta : U \to \im\theta$ by $u\mapsto u\theta$ which is a homomorphism because it is the restriction of $\theta$ to a subspace. The kernel of $\overline\theta$ is trivial as $\ker\overline\theta = \ker\theta\cap U = \{0\}$. And $\im\theta = \im\overline\theta$ because if $w\in\im\theta$ then $w=v\theta$ for some $v\in V$ and $v=k + u$ for some $u\in U$ and $k\in\ker\theta$. We then have
\[
	w = v\theta = (u+k)\theta = u\theta = u\overline\theta \in \im\overline\theta.
\]
Finally, by the first isomorphism theorem we obtain
\[
	U \simeq U/\ker\overline\theta \simeq \im\overline\theta = \im\theta.
\]
\end{proof}



% ---------------------------------------------------------------------------------
% LECTURE 7
% ---------------------------------------------------------------------------------



\section{Lecture 7}

\subsection{The Group Algebra and Irreducible Modules: Part II}


\begin{definition}
	For $V$, a $\CG$-module, $U$ an irreducible $\CG$-module, $U$ is called a composition factor for $V$ if $V$ has a $\CG$-submodule isomorphic to $U$.
\end{definition}

\begin{lemma}\label{Lem:decomp}
	Let $V$ be a $\CG$-module, suppose we have a finite decomposition into into a direct sum of irreducible $\CG$-modules:
	\[
	V = \bigoplus_{i=1}^n U_i
	\]
	Then, if $0<U<V$ is an irreducible, $\CG$-submodule of $V$, then $U\simeq U_i$ for some $1\leq i\leq n$.
\end{lemma}

\begin{proof}
	Consider the maps $\pi_i:U\rightarrow U_i$, since each $\pi_i:V\rightarrow U_i$ is a $\CG$-homomorphism, and since $U$ is non-trivial, it is clear that not all the $\pi_i$ are the zero map, let $\pi_j$ be such a projection, then by Schur, $\pi_j:U\rightarrow U_j$ is a $\CG$-isomorphism.
\end{proof}
\begin{theorem}
	Let the following be a decomposition of the regular representation into irreducible $\CG$-modules:
	\[
	\CG = \bigoplus_{i = 1}^n U_i
	\]
	Then every irreducible $\CG$-module is isomorphic to $U_i$ for some $1\leq i\leq n$
\end{theorem}
\begin{proof}
	Let $W$ be an irreducible $\CG$-module, let $0\neq w\in W$, then $wG = \{wr:r\in \CG\}$ is a $\CG$-submodule of $W$. However, $W$ is irreducible, so $W=wG$. Define $\theta:\CG\rightarrow W$ by $r\theta = wr$ for $r\in \CG$, then $\theta$ is a linear $\CG$-homomorphism by \ref{Lem:FG} and $\im\theta=W$. So $\CG = \ker\theta\oplus U$, where $U\simeq \im \theta\simeq W$ by \ref{Lem:FIT}. Now $W$ is irreducible, so $U$ is irreducible, so by \ref{Lem:decomp}, $U\simeq U_i\simeq W$ for some $1\leq i\leq n$.
\end{proof}

Now for a finite group $G$, we can characterise all irreducible $\CG$-modules by decomposing $\CG$.

\begin{example}
	Let $G = C_3 = \langle a|a^3=1\rangle$, then let $\omega = e^{\frac{2\pi i}{3}}$ be a primitive 3rd root of unity and let the following be elements of $\CG$:
	\[
	v_1 = 1+a+a^2,\qquad v_2 = 1+\omega^2 a+\omega a^2,\qquad v_3 = 1+\omega a +\omega^2 a^2
	\]
	We have $(v_1)a = v_1$, $(v_2)a = \omega+a+\omega^2a^2=\omega v_2$ and $(v_3)a = \omega^2+a+\omega a = \omega^2v_3$
	Therefore, $V_i = \langle v_i\rangle<\CG$ for $i=1,2,3$ are all irreducible submodules of $\CG$.
	
	Furthermore, it is clear that $\{v_1,v_2,v_3\}$ forms a basis for $\CG$, hence $\CG=V_1\oplus V_2\oplus V_3$.
\end{example}

\begin{example}
Let $G = D_6 = \langle a,b|a^3=1,a^b=a^{-1}\rangle$ be the dihedral group of order 6, and let $\omega = e^{2\pi i/3}$ be a third root of unity. Define $v_i \in \C G$ by $v_ia = \omega^i v_i$, then
\begin{align*}
    v_0 &= 1 + a + a^2,\\
    v_1 &= 1 + \omega^2 a + \omega a^2,\\
    v_2 &= 1 + \omega a + \omega^2 a^2.
\end{align*}
Further define $u_i = bv_i$ for all $i=0,1,2$. Then it follows that $\langle v_i \rangle$ and $\langle w_i \rangle$ are $\C \langle a \rangle$-modules for all $i$. Further, a short computation shows that
\[
    \langle v_0,u_0 \rangle, \quad \langle v_1, u_2 \rangle, \quad \langle v_2, u_1 \rangle
\]
are $\C\langle b \rangle$-modules. Hence, we see that
\begin{align*}
    U_1 &= \langle v_0 + u_0 \rangle,\\
    U_2 &= \langle v_0 - u_0 \rangle,\\
    U_3 &= \langle v_1, u_2 \rangle,\\
    U_4 &= \langle v_2, u_1 \rangle,
\end{align*}
are irreducible $\C G$-submodules. It is clear that $U_3 \simeq U_4$ via the map that sends $v_1 \mapsto u_1$ and $u_2 \mapsto v_2$. Finding the representations of each $U_i$ we have
\begin{align*}
    \rho_1 &: a \mapsto (1), \quad b \mapsto (1),\\
    \rho_2 &: a \mapsto (1), \quad b \mapsto (-1),\\
    \rho_3 &: a \mapsto \begin{pmatrix*}
        \omega & 0\\0 & \omega^{-1}
    \end{pmatrix*},\quad b \mapsto \begin{pmatrix*}
        0 & 1\\1 & 0
    \end{pmatrix*}.
\end{align*}
Hence $U_1$, $U_2$, and $U_3$ are non-isomorphic irreducible $\C G$-submodules. Moreover, we have
\[
    \C G = \underbrace{U_1}_{\text{trivial}} \oplus \;U_2 \oplus \underbrace{U_3 \oplus U_4}_{\text{isomorphic}}.
\]
\end{example}




\subsection{The Vector Space of $\CG$-homomorphisms}

\begin{definition}
	Let $V,W$ be $\CG$-modules, then define $H = \hom_\CG(V,W) = \{\theta:V\rightarrow W: \theta $ is a $\CG$ homomorphism$\}$.
\end{definition}
We can then define addition and scalar multiplication operations on this set for $\theta,\phi\in H$, $\lambda\in \C$ as follows:
\[v(\theta+\phi):=v\theta+v\phi,\qquad v(\lambda\theta)=\lambda(v\theta)\]
\begin{lemma}
	The space $H$ with the operations defined above is a vector space over $\C$.
\end{lemma}
\begin{proof}
	Clear from the definition, note that the zero homomorphism is: $v0=0_W$.
\end{proof}
\begin{lemma}
	For irreducible $\CG$-modules, $V,W$,
	\[
	\dim(H) = \begin{cases}
	1 & V\simeq W\\
	0 & V\not\simeq W
	\end{cases}
	\]
\end{lemma}
\begin{proof}
	If $V\not\simeq W$, then it follows that $\theta=0$ for all $\theta\in H$ by Schur.
	
	Now, suppose $V\simeq W$ and let $\theta:V\rightarrow W$ be an isomorphism. Now, let $\phi\in H$, we have that $\phi\theta^{-1}:V\rightarrow V$ is a homomorphism, so by Schur, $\phi\theta^{-1}=\lambda \identity_V$, so for $v\in V$, $v\phi\theta^{-1}=\lambda v\implies v\phi\theta^{-1}\theta=v\phi=(\lambda v)\theta=\lambda(v\theta)$, therefore $\phi=\lambda \theta$, so $\dim H=1.$
\end{proof}
\begin{proposition}
	Given $V,V_1,V_2,W,W_1,W_2$ $\CG$-modules, 
	\begin{enumerate}
		\item $\dim\hom_\CG(V,W_1\oplus W_2)=\dim\hom_\CG(V,W_1)+\dim\hom_\CG(V,W_2)$
		\item $\dim\hom_\CG(V_1\oplus V_2,W)=\dim\hom_\CG(V_1,W)+\dim\hom_\CG(V_2,W)$
	\end{enumerate}
\end{proposition}
\begin{proof}
	I will prove the first statement and the conversion of the proof ot the second statement is left as a simple exercise.
	
	Firstly, we define the projection homomorphisms: $\pi_i:W_1\oplus W_2\rightarrow W_i$ defined by $(w_1+w_2)\pi_i=w_i$. Now, if $\theta\in \hom_\CG(V,W_1\oplus W_2)$, we have that $\theta\pi_i\in \hom_\CG(V,W_i)$. 
	
	Define a linear map $f:\hom_\CG(V,W_1\oplus W_2)\rightarrow \hom_\CG(V,W_1)\oplus\hom_\CG(V,W_2)$ by $\theta\mapsto \theta\pi_1\oplus\theta\pi_2$ then it is clear that the following map is an inverse of $f$: \[
	v(\theta_1\oplus\theta_2)f^{-1}=v\theta_1+v\theta_2
	\]
	therefore we have an isomorphism of vector spaces, $\hom_\CG(V,W_1\oplus W_2)\simeq \hom_\CG(V,W_1)\oplus\hom_\CG(V,W_2)$, giving us $\dim\hom_\CG(V,W_1\oplus W_2)=\dim\hom_\CG(V,W_1)\oplus\hom_\CG(V,W_2) = \dim\hom_\CG(V_1,W)+\dim\hom_\CG(V_2,W)$
\end{proof}
\begin{corollary}
	For $V_1,\dots V_n$, $W_1,\dots W_m$, if $V = \bigotimes_{i=1}^nV_i$ and $W = \bigotimes_{i=1}^mW_i$
	\[
	\dim H = \sum_{i=1}^{n}\sum_{j=1}^m\dim\hom_\CG(V_i,W_j)
	\]
\end{corollary}
\begin{corollary}\thlabel{number-iso-to}
	Suppose $$V = \bigoplus_{j = 1}^n U_j$$ Then for any irreducible module $W$ \[\dim \hom(V, W) = \dim \hom(W, V) = |\{j \mid U_j\cong W\}|\]
\end{corollary}
\begin{example}
	Recall that $$\C D_3 = \bigoplus_{i = 1}^4 U_i$$ where $U_3 \cong U_4$. Then $$\dim \hom(\C D_3, U_3) = 2$$
\end{example}




%-----------------------------------------------------------------------------
% LECTURE 8
%-----------------------------------------------------------------------------


\section{Lecture 8}
\subsection{The Dimension of $\hom(U, V)$}
We continue the discussion last lecture about $\hom(U, V)$.

\begin{lemma}
	Let $V, W$ be two $\CG$-modules such that $\hom(V, W)$ is nonzero. Then $V$ and $W$ share a composition factor.
\end{lemma}
\begin{proof}
	Supopse we have a morphism $\theta: V \rightarrow W$. Then there exists some $v\in  V$ such that $v\theta \neq 1$. Now $v$ is contained in some irreducible submodule, say $V_0$. Then $V_0\theta \cong V_0$.
\end{proof}

\begin{lemma}\thlabel{hom-dim}
	Let $U$ be a $\CG$-module. Then $$\dim\hom(\CG, U) = \dim U$$. 
\end{lemma}
\begin{proof}
	Fix a basis $\{u_1,\ldots,u_n\}$ for $U$ and define $\varphi_i: \CG \rightarrow U$ as $r\mapsto u_ir$. We claim the $\varphi_i$ form a basis for $\hom(\CG, U)$. Indeed, let $\varphi\in \hom(\CG, U)$ and suppose \[(1)\varphi = \sum_{i = 1}^n \lambda_i u_i  \] Then for all $r\in \CG$ we have \[(r)\varphi = (1)\varphi r = (\sum_{i = 1}^n \lambda_i u_i)r = \sum_{i = 1}^n \lambda_i u_ir = \sum_{i = 1}^n \lambda_i (r)\varphi_i\] where the first equality follows from \thref{module-hom}
\end{proof}

\begin{theorem}
	Suppose $$\CG = \bigoplus_{j = 1}^n V_j$$ and $U$ is an irreducible module. Then the number of $j$ such that $V_j\cong U$ is exactly $\dim U$
\end{theorem}
\begin{proof}
	Combine \thref{hom-dim} and \thref{number-iso-to}
\end{proof}

\begin{example}
	Recall that $$\C D_3 = \bigoplus_{i = 1}^4 U_i$$ where $U_3 \cong U_4$ but $U_1 \ncong U_2$. Then $U_1$ and $U_2$ occur once whereas $U_3$ occurs twice, consistently with the theorem.
\end{example}

\begin{theorem}
	Let $V_1,...,V_n$ denote a complete set of irreducible modules that are pairwise non-isomorphic. Then \[\sum_{i = 1}^n (\dim V_i)^2 = |G|\]
\end{theorem}
\begin{proof}
	Suppose $$\CG = \bigoplus_{j = 1}^N U_j$$ where for each $V_i$ there are exactly $\dim V_i$ of the $U_j$ isomorphic to $V_i$. Thus we have $$|G| = \dim \CG = \sum_{i = 1}^N \sum_{j = 1}^{\dim V_i} \dim U_j = \sum_{i = 1}^N \sum_{j = 1}^{\dim V_i} \dim V_i = \sum_{i = 1}^n (\dim V_i)^2$$
\end{proof}

Observe that $\CG$ always has a trivial submodule, namely the module spanned by $\sum_{g\in G} g$.

\begin{example}
	Note that $|D_3| = 6$ and $6 = 1^2 + 1^2 + 2^2$. This is the only way; indeed, if all irreducible submodules are of dimension 1, then $D_3$ would be abelian, which is obviously false.
\end{example}










% -----------------------------------------------------
% LECTURE 9
% ----------------------------------------------------


\section{Lecture 9}


\subsection{Group Theoretic Diversion}


Suppose $G$ is a group. We define a equivalence relation on $G$ called \textbf{conjugacy} by
\[
    x \sim y \iff y = x^g = g^{-1}xg, \quad \text{for some }g \in G.
\]
The equivalence class
\[
    x^G = G^{-1}xG = \{g^{-1}xg \mid g \in G\},
\]
is called the \textbf{conjugacy class} of $x$.

\begin{lemma}
Every group is a union of conjugacy classes and distinct classes are disjoint.
\end{lemma}

\begin{proof}
Every equivalence relation on a set corresponds to a partition of said set.
\end{proof}

\begin{example}
For any group $G$, $1^G = \{1\}$ is a conjugacy class in $G$. More generally, if $x \in Z(G)$ then $xg = gx$ for all $g \in G$; from which it follows that $x^G = \{x\}$.
\end{example}

\begin{example}
Let $G = D_6$ the dihedral group of 6 elements, generated by the elements $a,b$. Then $a^G = \{a,a^2\}$, and $b^G = \{b,ab,a^2b\}$. Hence $D_6 = 1^G \amalg a^G \amalg b^G$.
\end{example}

\begin{example}
If $G$ is an abelian group, then for all $x \in G$, $x^G = \{x\}$. This follows from a previous example as $G$ is abelian if, and only if, $G = Z(G)$.
\end{example}

\begin{lemma}
Suppose that $x,y \in G$ with $x \sim y$, then $x^n \sim y^n$ for all $n \in \N$. In particular, $|x| = |y|$.
\end{lemma}

\begin{proof}
As $x \sim y$ there exists $g \in G$ such that $x = g^{-1}yg$. By induction, it follows that $x^n = g^{-1}y^ng$ which shows that $x^n \sim y^n$. To see that the orders are equal, note that $x^n = 1$ if, and only if $g^{-1}y^n g = 1$. Hence $y \in 1^G = \{1\}$ and so $y^n = 1$.
\end{proof}

Suppose $x \in G$. Define the \textbf{centraliser} of $x$ in $G$ to be the set
\[
    C_G(x) = \{g \in G \mid xg = gx\} = \{g \in G \mid x^g = x\},
\]
i.e. the set of $g \in G$ which fix $x$ under conjugation. It is clear that $C_G(x) \le G$

\begin{theorem}[Orbit-stabiliser]
Suppose $G$ is a finite group and $x \in G$. Then $|x^G| = |G:C_G(x)| = |G|/|C_G(x)|$, and in particular $|x^G| \mid |G|$.
\end{theorem}

\begin{proof}
First we have the chain of equivalences:
\begin{align*}
    g^{-1}xg = h^{-1}xh &\iff hg^{-1}x = xhg^{-1}\\
    &\iff hg^{-1} \in C_G(x)\\
    &\iff C_G(x)g = C_G(x)h.
\end{align*}
Hence let $\Lambda$ denote the set of right cosets of $C_G(x)$ in $G$, and define the function
\begin{align*}
    f : x^G &\to \Lambda\\
    g^{-1}xg &\mapsto C_G(x).
\end{align*}
Then $f$ is well-defined by the previous working. Moreover, the previous working also shows that $f$ is injective, and it is clearly surjective. Thus $|x^G| = |G : C_G(x)|$.
\end{proof}

Observe that
\begin{align*}
    |x^G| = 1 &\iff g^{-1}xg = x \quad \forall g \in G\\
    &\iff xg = gx \quad \forall g \in G\\
    &\iff x \in Z(G).
\end{align*}


\begin{theorem}[Class equation]
Let $G$ be a finite group and suppose $G = \coprod_i x_i^G$. Then
\[
    |G| = |Z(G)| + \sum_{x_i \notin Z(G)} |x_i^G|,
\]
where $|x_i^G| = |G : C_G(x_i)|$ and both components divide $|G|$.
\end{theorem}

\begin{proof}
As $G$ is a disjoint union of conjugacy classes, we have
\[
    |G| = \abs{\coprod_i x_i^G} = \sum_i |x_i^G|.
\]
Finally use the fact that $x \in Z(G)$ if, and only if, $|x^G| = 1$. The fact $|x_i^G| = |G : C_G(x_i)|$, and both components divide $|G|$ follow from the orbit-stabiliser theorem. 
\end{proof}










% -----------------------------------------------------
% LECTURE 10
% ----------------------------------------------------


\section{Lecture 10}


\subsection{Class Sums}

\begin{definition}
Let $C$ be a conjugacy class of group $G$. We define a class sum to be the sum off all elements in our conjugacy class denoted
\[
	\overline C = \sum_{g\in C}g.
\]
\end{definition}

We now note the importance of these sums in the following theorem.

\begin{theorem}
The class sums $\overline C_1, \ldots, \overline C_l$ form a basis for $Z(\mathbb{C}G)$.
\end{theorem}

\begin{proof}
	We first note that the class sums are closed under conjugation and therefore are elements of the center. Now suppose $\sum_{i=1}^l\lambda_i\overline C_i = 0$. Since conjuacy classes are pairwise disjoint we obtain $\lambda_i = 0$ for all $1\leq i \leq l$ so the $\overline C_i$ are linearly independent. We now need to show they span $Z(\mathbb{C}G)$. Pick some $r = \sum_{g\in G}\lambda_gg \in Z(\mathbb{C}G)$ and $h\in G$. Since $r$ is central $r^h = r$ and therefore
	\[
		\sum_{g\in G}\lambda_g g^h= \sum_{g\in G}\lambda_g g.
	\]
	Since the $h\in G$ was arbitrary we see that if $x\sim y$ then $\lambda_x = \lambda_y$ so $r$ can be written as a sum of class sums.
\end{proof}

Note now that we immediately obtain an important result. The dimension of $Z(\mathbb{C}G)$ is exactly the number of conjugacy classes of $G$.

\begin{example}
Let $G = S_3$. The conjugacy classes of $S_3$ are given by
\[
	\{\varepsilon\}, \{(1\;2), (2\;3), (1\;3)\}, \{(1\;2\;3), (1\;3\;2)\}.
\]
Then $Z(\mathbb{C}G)$ has dimension 3.
\[
	Z(\mathbb{C}G) = \langle 1, (1\;2) + (2\;3) +(1\;3), (1\;2\;3) + (1\;3\;2) \rangle.
\]
\end{example}



\subsection{Characters}

\begin{definition}
If $A = (a_{ij})$ is an $n\times n$ matrix, then the trace of $A$ is the sum of the diagonal elements.
\[
	\tr A = \sum_{i=1}^n a_{ii}
\]
\end{definition}

We now present some basic properties of the trace.

\begin{lemma}
Let $A,B,T\in M_{n}(\mathbb{C})$ and $T$ be invertable.
\begin{enumerate}[label=\emph{(\Roman*)}]
	\item $\tr(A+B) = \tr(A) + \tr(B)$
	\item $\tr(AB) = \tr(BA)$
	\item $\tr(T^{-1}AT) = \tr(A)$
\end{enumerate}
\end{lemma}

\begin{proof}
(I) and (II) follow from the fact that if $C = A + B$ then $c_{ii} = a_{ii} + b_{ii}$ and if $D = AB$ then $d_{ii} = a_{ii}b_{ii}$. (III) follows from an application of (II)
\[
	\tr(T^{-1}AT) = \tr(T^{-1}(AT)) = \tr((AT)T^{-1}) = \tr(A).
\]
\end{proof}

Now that we have the machinery to describe them we define a character.

\begin{definition}
	Let $V$ be a $\mathbb{C}G$ module with basis $B$. The character of $V$ is the map $\chi: G\to\mathbb{C}$ defined by
\[
	\chi(g) = \tr [g]_B.
\]
\end{definition}

Note that by property (III) of the trace that a character is independent of basis and so we uniquely associate a character to a $\mathbb{C}G$ module. In the following lemma we see that we can associate a character to a module up to isomorphism.

\begin{lemma} ~
\begin{enumerate}[label=\emph{(\Roman*)}]
	\item Isomorphic $\mathbb{CG}$ modules have the same character.
	\item If $x,y\in G$ then $x\sim y \implies \tr[x] = \tr[y]$.
\end{enumerate}
\end{lemma}


\begin{proof}
(I) If $V$ and $W$ are isomorphic $\mathbb{C}G$-modules then there exists bases $B_1, B_2$ such that $[g]_{B_1} = [g]_{B_2}$ for all $g\in G$.

	(II) If $x\sim y$ then $x = g^{-1}yg$ for some $g\in G$. Then $\tr[x] = \tr[g^{-1}yg] = \tr[y]$.
\end{proof}

\begin{example}
Let $G = S_3 = \langle a=(1\;2),b=(1\;2\;3)$ and let $V = \langle v_1,v_2,v_3 \rangle$ be the permutation module of $G$. We then have representations
\[
	[a] = \begin{pmatrix} 0&1&0\\ 1&0&0\\ 0&0&1 \end{pmatrix}, \quad
	[b] = \begin{pmatrix} 0&1&0\\ 0&0&1\\ 1&0&0 \end{pmatrix}.
\]
and therefore character of the permutation module $\chi$ satisfies
\[
	\chi(a) = 1, \quad \chi(b) = 0.
\]
Note that because characters are constant on conjugacy classes, to specify a character of a group we only need to define it on the group's classes. As a result here we know that $\chi$ will take any 2-cycle to 1 and any 3-cycle to 0.
\end{example}

\begin{definition}
The dimension of a character $\chi$ is $\chi(1)$.
\end{definition}

The character of a one-dimensional $\mathbb{C}G$-module is called a linear character. By Schur's lemma for each $g\in G$ there exists some $\lambda_g$ such that $vg = \lambda_gv$ for all $v\in V$. Thus a linear character will take $v\mapsto \lambda_g$.

\begin{lemma}
	Every linear character is a homomorphism from $G$ to $\mathbb{C}^*$; The multiplicative group of $\mathbb{C}$.
\end{lemma}

\begin{proof}
Suppose $\chi$ is a linear character of $G$. Note that $\chi(e) = \lambda_e = 1$ and if $g\in G$ then $\chi(g)\neq 0$ as $\chi(g)\chi(g^{-1}) = \chi(e) = 1$. Now pick $g,h\in G$. Then 
\[
	\chi(gh) = \lambda_{gh} = \lambda_g\lambda_h = \chi(g)\chi(h).
\]
\end{proof}

Note that the multiplicative properties of a linear character don't hold for all characters in general. That is for matricies $A,B$, $\tr(AB) \neq \tr(A)\tr(B)$ in general.

We now summarise some properties of a character.

\begin{lemma}\label{Lem:Charm}
Let $V$ be a $\mathbb{C}G$-module with character $\chi$ and let $g\in G$ with $|g|=m$.
\begin{enumerate}[label=\emph{(\Roman*)}]
	\item $\chi(1) = \dim V$,
	\item $\chi(g)$ is a sum of $m$-th roots of unity,
	\item $\chi(g^{-1}) = \overline{\chi(g)}$,
	\item $\chi(g)$ is real if $g\sim g^{-1}$.
\end{enumerate}
\end{lemma}

\begin{proof}
(I) The representation of 1 will be the identity matrix. The trace of which is the dimension of the representation.

(II) There is a basis of $V$ in which the representation of $g$ is a diagonal matrix of $m$-th roots of unity.

(III) Note by the previous part we can write $\chi(G) = \sum{i=1}^n\omega_i$ where $\omega_i$ is an $m$-th root of unity. The representation of $g^{-1}$ will be the same diagonal matrix but with the inverse of each root of unity so $\chi(g^{-1}) = \sum_{i=1}^n \omega_i^{-1}$. However the inverse of a root of unity is its conjugate and the sum on conjugates is the conjugate of the sum so
\[
	\chi(g^{-1}) = \sum_{i=1}^n\overline{\omega_i} =\overline{\chi(g)}.
\]

(IV) This follows from (III).
\end{proof}

\begin{definition}
The character of the trivial module is called the trivial character. It sends all elements of $G$ to 1. 
\end{definition}

\section{Lecture 11}
\subsection{Restrictions on Characters}
We start with a quick corollary to Lemma 10.11
\begin{corollary}
	Let $V$ be a $\mathbb{C}G$-module of dimension $n$ with character $\chi$ and let $g\in G$, suppose $|g|=2$, then $\chi(g)\in \Z$ and $\chi(g)\equiv \chi(1)\mod 2$
\end{corollary}
\begin{proof}
	By \ref{Lem:Charm}, we have that $\chi(g) = \sum_{i=1}^n\omega_i$ for each $\omega_i$ a 2th-root of unity, that is, $\omega_i = \pm 1\equiv 1\mod 2$, therefore it is clear that $\chi(g)\in \Z$. We can also calculate, 
	\[
	\chi(g) = \sum_{i=1}^n\omega_i\equiv \sum_{i=1}^n1 \bmod 2 \equiv n\bmod 2 \equiv \chi(1)\bmod 2
	\]
	as $\chi(1)=n$ by \ref{Lem:Charm}.
\end{proof}
Note, that we can assign a character to a representation even more naturally as we assigned one to a module by $\chi = \tr\circ \rho$
\begin{theorem}
	Let $\phi:G\rightarrow GL(n,\C)$ be a representation of $G$  and let $\chi$ be the representation of $\phi$.
	\begin{enumerate}[label=\emph{(\Roman*)}]
		\item  For $g\in G$, $|\chi(g)| = \chi(1)\iff g\rho = \lambda I_n$ for some $\lambda\in \C$
		\item $\ker \rho = \{g\in G:\chi(g) = \chi(1) = n\}$
	\end{enumerate}
\end{theorem}
\begin{proof}
	For (I), we firstly prove the reverse implication. Let $g\in G$ and let $m = |g|$, then $I_n = 1\rho = g^m\rho = \lambda^mI_n \implies \lambda^m = 1$, that is, $\lambda$ is an $m$-th root of unity. Since $\chi(g) = n\lambda$, we have $|\chi(g)| = n|\lambda| = n = \chi(1)$.
	
	Now, instead suppose that $|\chi(g)| = \chi(1) = n$. By \ref{Lem:Charm}, we can write $\chi(g) = \sum_{i=1}^n\omega_i$ where $\omega_i$ are $m$-th roots of unity. Therefore $|\chi(g)| = |\sum_{i=1}^n\omega_i| = n = \sum_{i=1}^n|\omega_i|$, by strict convexity of $\C$ with the absolute value, this is only true if each $\omega_i$ is a positive real multiple of every other $\omega_j$, however, all have length 1, therefore all the $\omega_i$ are equal. So $\chi(g) = \omega_1I_n$. 
	
	(II) Suppose $g\in \ker\rho$, then $\chi(g) = \tr(I_n) = n = \chi(1)$. Now, instead, suppose that $\chi(g) = \chi(1) = n$, then by (I), $g\rho = \lambda I_n$ for some $\lambda\in \C$. Therefore $\chi(1) = \chi(g) = \lambda\chi(1)$, so $\lambda = 1$ and $g\rho = I_n\implies g\in\ker\rho$.
\end{proof}

\begin{definition}
	Let $\chi$ be a character of $G$. We define the kernel of $\chi$ by:
	\[
	\ker\chi = \{g\in G:\chi(g) = \chi(1)\}
	\]
	If $\rho$ is a representation of $G$ and $\chi$ is its character, we get $\ker\chi = \ker\rho$. Therefore $\ker\chi\unlhd G$. And $\rho$ is faithful if and only if $\ker\chi = \{1\}$.
\end{definition}
\begin{example}
	$G = D_6$

\begin{tabular}{l|l|l|l|l|l|l|}
	& $1$ & $a$  & $a^2$ & $b$  & $ab$ & $a^2b$  \\ 
	\hline
	$\chi_1$ & $1$ & $1$  & $1$   & $1$  & $1$  & $1$     \\ 
	\hline
	$\chi_2$ & $1$ & $1$  & $1$   & $-1$ & $-1$ & $-1$    \\ 
	\hline
	$\chi_3$ & $2$ & $-1$ & $-1$  & $0$  & $0$  & $0$     \\
	\hline
\end{tabular}

\end{example}

\end{document}
